%% 
%% Copyright 2007, 2008, 2009 Elsevier Ltd
%% 
%% This file is part of the 'Elsarticle Bundle'.
%% ---------------------------------------------
%% 
%% It may be distributed under the conditions of the LaTeX Project Public
%% License, either version 1.2 of this license or (at your option) any
%% later version.  The latest version of this license is in
%%    http://www.latex-project.org/lppl.txt
%% and version 1.2 or later is part of all distributions of LaTeX
%% version 1999/12/01 or later.
%% 
%% The list of all files belonging to the 'Elsarticle Bundle' is
%% given in the file `manifest.txt'.
%% 

%% Template article for Elsevier's document class `elsarticle'
%% with numbered style bibliographic references
%% SP 2008/03/01

\documentclass[preprint,12pt]{elsarticle}
\usepackage{natbib}
% \usepackage{conference}
\usepackage{latexsym}
% \usepackage[utf8]{inputenc}
\usepackage[utf8x]{inputenc}
% \usepackage{textcomp}
% \usepackage[english]{babel}
\usepackage{amsfonts,amsmath,amssymb}
% \usepackage{amssymb}
\usepackage{graphicx}
% \usepackage{pstricks}
% \usepackage[numbers,super]{natbib}
% \usepackage{cite}
\usepackage{hyperref}
% \usepackage{lineno,hyperref}
% \modulolinenumbers[5]
\usepackage[varg]{txfonts}
% \usepackage{booktabs}       % professional-quality tables
% \usepackage{tabularx}
%\usepackage{enumerate}
%\usepackage{multicol}
%\usepackage{multirow}
%\pagenumbering{gobble}
%% Use the option review to obtain double line spacing
%% \documentclass[authoryear,preprint,review,12pt]{elsarticle}
%% Template article for Elsevier's document class `elsarticle'
%% with numbered style bibliographic references
%% SP 2008/03/01

%\documentclass[preprint,12pt]{elsarticle}

%% Use the option review to obtain double line spacing
%% \documentclass[authoryear,preprint,review,12pt]{elsarticle}

%% Use the options 1p,twocolumn; 3p; 3p,twocolumn; 5p; or 5p,twocolumn
%% for a journal layout:
%% \documentclass[final,1p,times]{elsarticle}
%% \documentclass[final,1p,times,twocolumn]{elsarticle}
%% \documentclass[final,3p,times]{elsarticle}
%% \documentclass[final,3p,times,twocolumn]{elsarticle}
%% \documentclass[final,5p,times]{elsarticle}
%% \documentclass[final,5p,times,twocolumn]{elsarticle}

%% For including figures, graphicx.sty has been loaded in
%% elsarticle.cls. If you prefer to use the old commands
%% please give \usepackage{epsfig}

%% The amssymb package provides various useful mathematical symbols
%\usepackage{amssymb}
%% The amsthm package provides extended theorem environments
%% \usepackage{amsthm}

%% The lineno packages adds line numbers. Start line numbering with
%% \begin{linenumbers}, end it with \end{linenumbers}. Or switch it on
%% for the whole article with \linenumbers.
%% \usepackage{lineno}

\journal{Journal de Mathematiques Pures et Appliquees}

\begin{document}

\begin{frontmatter}

%% Title, authors and addresses

%% use the tnoteref command within \title for footnotes;
%% use the tnotetext command for theassociated footnote;
%% use the fnref command within \author or \address for footnotes;
%% use the fntext command for theassociated footnote;
%% use the corref command within \author for corresponding author footnotes;
%% use the cortext command for theassociated footnote;
%% use the ead command for the email address,
%% and the form \ead[url] for the home page:
%% \title{Title\tnoteref{label1}}
%% \tnotetext[label1]{}
%% \author{Name\corref{cor1}\fnref{label2}}
%% \ead{email address}
%% \ead[url]{home page}
%% \fntext[label2]{}
%% \cortext[cor1]{}
%% \address{Address\fnref{label3}}
%% \fntext[label3]{}

%% \title{}

%% use optional labels to link authors explicitly to addresses:
%% \author[label1,label2]{}
%% \address[label1]{}
%% \address[label2]{}
\title{Atiyah's Physics-Mathematics Unification confirms Permanent Oscillatory Cosmology} %% \tnoteref{mytitlenote}}
%% \tnotetext[mytitlenote]{Fully documented equations and formulas are available in the github jupyter notebook on \href{https://github.com/LaGuer/hubble-radius}{hubble-radius}.}

%% Group authors per affiliation:
\author{Francis M. Sanchez\fnref{myfootnote}}
\address{Universite Paris Sud, Orsay}
\fntext[myfootnote]{Retired Professor}
\ead{hol137@yahoo.fr}
%% or include affiliations in footnotes:
% \author[mymainaddress,mysecondaryaddress]{Elsevier Inc}
\ead[url]{github.com/hol137/hubble-table}

\author[mysecondaryaddress]{Michel H. Grosmann\corref{mycorrespondingauthor}}
\ead{----@me.com}
\address[mymainaddress]{Universite de Strasbourg, 67000 Strasbourg}
\author[mysecondaryaddress]{Dominique Tassot\corref{mycorrespondingauthor}}
\ead{----@me.com}
\author[mysecondaryaddress]{Renee Veysseyre\corref{mycorrespondingauthor1}}
\ead{----@me.com}
\address[mymainaddress]{Ecole Centrale, 92 Chatenay Malabry}
\author[mysecondaryaddress]{Dominique Weigel\corref{mycorrespondingauthor}}
\ead{----@me.com}
\address[mymainaddress]{Universite Paris 6, 75 Paris}
\cortext[mycorrespondingauthor]{Retired Professor}
\cortext[mycorrespondingauthor1]{Agregee de mathematiques et professeur honoraire à l'Ecole centrale de Paris}


\address[mymainaddress]{Universite Paris Sud Orsay}


\begin{abstract}
%% Text of abstract
The Permanent Oscillatory Cosmology is confirmed by 71 formula giving the Hubble radius, with 7 correlating to $10^{-9}$. The computer shows that the best formula are obtained using the Atiyah constant and the number 137, the Eddington's integer part of the electric constant. This is conforms with Atiyah's testimony about the Physics-Mathematics unification and the central role of arithmetics in the unification process of both mathematics and physics. The identification with the Eddington statistical formula gives $G$, compatible with the $10^{-5}$ precise BIPM measurement and the $10^{-6}$ precise quasar non-Doppler Kotov period. The hypothesis of a computing Cosmos implies a $\pi$ rationalization process which validates the Wyler's theory and the Fermion Koide formula in the $10^{-9}$ domain. 
\end{abstract}

\begin{keyword}
\texttt{Quantum Physics}\sep Number theory \sep Holography \sep Crystallography \sep Cosmology
% \MSC[2010] 00-01\sep  99-00
\end{keyword}


\end{frontmatter}

%% \linenumbers

%% main text
\section{Introduction}
  
    From Hirzebruch's work \cite{Hirzebruch}, which revolutionized geometry and topology, Sir Michael Atiyah, Raoul Bott \cite{Bott} and Isadore Singer \cite{Singer} introduced the index theory, acclaimed by theoretical physicists \cite{Alvarez}. Following this path, on the advice of the physicist Gerard t'Hooft, Atiyah looked for the determination of the electrical constant $a \approx 137.035999085(21)$ \cite{Atiyah}.
    
    
    At the 2018 Heidelbergh Laureate Forum, he showed that the extrapolation of the Euler formula  $e^{2i\pi} = 1$ to the quaternions leads to the 'Atiyah constant' $\Gamma = \gamma a/\pi $. Meanwhile, he rehabilitated the Eddington \cite{Eddington} bare electrical constant, the prime number 137, and announced that the resolution of the Riemann conjecture appears as a "bonus". Moreover, the four forces would be connected to the four principal algebra, whose the octonion non-associative one would be tied to the gravitation constant $G$ in a future work \cite{Atiyah}. 

\label{}

%% The Appendices part is started with the command \appendix;
%% appendix sections are then done as normal sections
%% \appendix

%% \section{}
%% \label{}

%% If you have bibdatabase file and want bibtex to generate the
%% bibitems, please use
%%
%%  \bibliographystyle{elsarticle-num} 
%%  \bibliography{<your bibdatabase>}

%% else use the following coding to input the bibitems directly in the
%% TeX file.

%% \begin{thebibliography}{00}

%% \bibitem{label}
%% Text of bibliographic item
\section{The scope and method}
   Quite independently, the $G$ value was tied to the invariant Hubble radius $R$ in the \textit {Coherent Cosmology}. A computer analysis has shown that it is confirmed in the ppb domain ($10^{-9}$) by simple formula involving the Atiyah constant \cite{Sanchez}. The aim of this article is to confirm the unificaton process, both in mathematics and physics, according to the \textit {arithmetic} approach of Eddington, Hirzebruch and Atiyah.
   
   Our main hypothesis is that \textit{both} mathematics and physics standard model are incomplete, and that the so far unexplained measured adimensional constants (see Table 1) can be used as a guide for the overall arithmetics unification. Recall that the search for correlations between the measurements is the heart of the scientific method, as the history shows, through Dalton, Proust, Balmer, Mendeleiv, Mandel... In particular, the Atiyah constant  enters the core of Coherent Cosmology, the Topological Axis (Fig. 1), both in connection with the Higgs boson and the galaxy group radius, a crucial cosmic distance.
   
   
   In conformity with the Atiyah's physics-mathematics unification idea, this table 1 mixes physical adimensional constants \cite{Tanabashi} with pure mathematics constants. But among the later an important distinction is made. Only the whole nubers are considered as exact. For instance the Archimedes constant $\pi$ is refered only as 'exact', meaning one can uses it in a computer calculation, only if one defines an imprecision domain. From this argument, the Cosmos vastness has been justified by  quasi-continuous quantum holography, where the whole large numbers of Lucas-Mersenne \cite{Bastin} and Eddington \cite{Eddington} are central \cite{Sanchez}.  
      
   In particular, the fact that the Muon-Electrion mass ratio is measured to 10 ppb, while nobody knows the role of Muon in Nature, is very intriguing. We show here that this permits to definitely validate the empiric Koide formula, connected with a rehabilitation of Wyler's theory through, precisely,a $\pi$ rationalisation process. 
   
   While the Atiyah's work does not seem to give the $a$ value, nor the Riemann conjecture solution, he suggested \textit{there is a bridge between the octonion algebra and the sporadic groups} \cite{Atiyah1}. Now, the Topological Axis shows clearly the height-fold Bott periodicity, typical of octonion algebra, which is also present in the Periodic Table, and analysis has shown the central role of the Monster group order in Coherent Cosmology \cite{Sanchez}. This article will confirm this Atiyah's prediction, unifying two appently no-connected mathematical domains. Note that the sporadic domain has already been connected with the modular function \cite{Conway} \cite{Borcherds}. This will be central in this study.
       
     Moreover, the present article confirms that the totality of the 26 sporadic groups are involved, as well as, most unexpectedly, the \textit{multi-dimensional crystallography}. 

\section{The cosmic liaison between $a$ and the weak mixing angle}

Thus, the physical parameters would be mathematical constants of an unknown arithmetical domain. So, their ``fine-tuning'' is not due to hazard in a Disparate Multiverse, but are of mathematical origin in a single Cosmos unifying coherent universes. The main result of a preceding study is that the Cosmos volume, with length unit the Hydrogen radius $r_H$, involves $a^a$, showing that \textit{$a$ is an optimal computation base}  \cite{Sanchez}:

\begin{equation}
    (4\pi /3) (R_C/r_H)^3 \approx a^a/\pi \approx (1/ln2)^{\sqrt{pH}} \approx (13/3)^{p/4} \approx (1/sin^2\theta)^{n/4} 
\end{equation}

where $p$, $H$ and $n$ are the proton-electron, hydrogen-electron and neutron-electron mass ratios, and 13/3 the fraction associated to the decomposition 16 = 13 + 3 \cite{Sanchez1}. This corresponds to the value $sin^2\theta \approx 0.231235$, compatible with the measured weak mixing angle 0.21322(4) \cite{Tanabashi}. 


Note that the presence of $ln2$ invoves information theory \cite{Shannon}.
  
\section{The Rehabilitation of Wyler\'s theory}
The presence of an excess $\pi$ in the above formula suggests that $\pi$ is also a computation base for the cosmos.This is indeed the case in the even Riemann series.


Atiyah did not consider this computation point of view, but insisted on the analogy of his procedure with that of Archimedes for calculing $\pi$ \cite{Atiyah}. But, \textit {in the hypothesis that the cosmos is a computer}, the cosmos cannot use the mathematical Archimedes constant $\pi$, which is an idealisation, \textit {otherwise any time calculation would be infinite}. Its decomposition is an unresolved problem, but the first terms are : $3, 7, 16, -293.634$, where the fourth term, hightly singular, is so close (3 ppm) to $1 + n/2\pi$, where $n$ is the neutron/electron mass ratio.

\begin{equation}
\pi : 3, 7, 16, -(1+n/2\pi)
\end{equation}


In the famous Wyler formula \cite{Wyler} 

\begin{equation}
(3\sqrt a/4))^8 = 120 \times \pi_W^{11}
\end{equation}


implicitely tied to the 11D supergravity space, the development of  $\pi_W$ shows an analogy with the above, apart the insertion of \textit {the singular prime 163} :

\begin{equation}
\pi_W : 3, 7, 16,- 163/2, -(1+n/4\pi) ~~~~\Rightarrow ~~~~    a \approx 137.03599908399
\end{equation}


This number 163 is the last of the Heegner-Stark numbers \cite{Stark}. 


Moreover, according to Atiyah \cite{Atiyah1}, an approximation of $\pi$ appears directly in $(a^2-137^2)^{1/2} = \pi_{a,137} : 3, 7, 10, a_s$, , where the forth term is very close to the inverse strong coupling constant $1/a_s \approx 0.1179(10)$ \cite{Tanabashi}. The proton-electron mass ratio of Wyler \cite{Wyler} is the simple formula 

\begin{equation}
 p_W = 6\pi^5    
\end{equation}

which is \textit {the product of the area of a cube of side $\pi$ with its volume}. Taking the above value $\pi_W$, this gives $p_{W, a, 137} \approx 1834$, which is of central pertinence in cosmology: indeed it connects both with $R$ and $R_1$, the one-electron universe radius \cite{Sanchez} (table 1).

\section{The Central Role of the Modular Number 744}

The \textit{Ramanujan quasi-whole number} $N_R = exp(\pi\sqrt(163))$, is tied to the Dedekind eta function, which plays a role in bosonic string theory \cite{Apostol}\cite{Lovelace}, wholly rehabilitated by the Topological Axis (Fig.1).

This large number is also tied to the modular function $j$, whose Fourier series shows linear combinations of dimensions of the irreductive representations of the Monster group. With $q = 2\pi x$ :

\begin{equation}
j(x) = 1/q + 744 + 196884 q + ...
\end{equation}

In particular 196884 = D + 1 where D = 196883 is the Monster group order. This was called the Monster Moonshine \cite{Conway}. It was shown that string physics makes a bridge between two separated mthematical domains \cite{Borcherds}. 


But the q-independent number number 744 is unexplained. It is related to the above fraction 13/3 by the relation involving the Monster and the bosonic string dimension 26:

\begin{equation}
  O_M^{1/26^2} \approx 744^{1/6^2}\approx (1+1/d_e)^{1/2}   
\end{equation}

where $d_e \approx 1.001159$ is the electron magnetic moment excess.

One  remarks it is 744 = (3/2)496, where 496 is the dimension of the superstring gauge group SO32, a necessary conditions for a superstring theory to make sense \cite{Green}.


But 10-dimensional string theory is the version of the theory that uses octonions algebra \cite{Schlay}. So, it seems that the Atiyah's conjecture was correctly prophetic. The present study confirms this connection octonion-sporadics from cosmology.

\section{The Decisive Role of the term $a^a$ }

Moreover, the product of the 6 pariah sporadic groups is directly tied with $a^a$ and $F/a$ (table 7), $F$ being the ratio Fermi/electron. 

Indeed, the above canonical number $a^a$  is also very close to the Lucas-Lehmer term $S_9 = g_3^{2^9}$, where $ g_3 = 2 + \sqrt(3)$ is the generator of quasi-whole numbers. Now, the Lucas-Mersenne Large Number $2^{127} - 1$ plays a central role in Coherent Cosmology \cite{Sanchez}. It is prime because it is a divisor of the number $S_{125}$, which appears to connect also in cosmology (first formula of table 2).

Now, $a^a$ connects also directly with the famous Ramanujan quasi-whole number $N_R = exp(\pi \sqrt(163))$, tied to the above Heegner-Stark number 163, which shows dramatic correlations:



\begin{equation}
lnR_N = \pi \sqrt(163)  \approx lna \times ln\tau  \approx  lnp \times ln\mu
\end{equation}

\begin{equation}
a^a \approx N_R^{\tau/\mu} 
\end{equation}

\begin{equation}
\tau/\mu   \approx g(1) \approx  2a_s
\end{equation}

where $g(k) = exp(2^k)/k$ is the Topological Function (Fig 1), while $p$, $\mu$ and $\tau$ are respectively the masses of Proton, Muon, and Tau by respect to the Electron one.  Now they seem to be related to Topological Axis Function $g(1)$ and the strong coupling $a_s$.


\section {The Koide-Wyler ppb relation}

While $\mu$ is measured to 0.1 ppm, $\tau$ is badly measured. The Koide relation \cite{Koide}, always unexplained, has shown correct predictability for the $\tau$ mass, proving \textit{the present standard particle theory is badly insufficient}. This relation writes in the most symmetrical way connecting with the above Wyler formula, in the following \textit{ppb formula, which confirms the specified  value}: $\mu = (Fa/\sqrt{pH})^{1/2}$\cite{Sanchez}.


\begin{equation}
(1 + \mu + \tau)/2 = (1 + \sqrt{\mu} + \sqrt{\tau})^2/3 = p_K \approx 6\pi_K^5 (1+(\mu / \tau)^2) ~~~~\pi_K: 3,7,16,-(2\times 137)^{2/3}
\end{equation}

Indeed, the fermions Mu and Tau are complete mystery in the standard model. However, Eddington predicted the tau fermion, 35 years before its surprising discovery, calling it ``heavy Mesotron'', with a right order prediction of its mass \cite{Eddington}.


\section{The rehabilitation of 137 from the electroweak coupling constant}

 This was very surprising because the Eddington theory, accused of pythagorism, was rejected. But he also predicted the importance of the $N_16 = 136$ elements of a symmetrical matrix 16 x 16, giving 137 by adding unity, whose pertinence is confirmed by the very precisely (0.1 ppm) measured electroweak coupling (inverse) factor 

\begin{equation}
a_w = (2\times137 \Gamma)^3
\end{equation}
 
Atiyah presented this number by the form $137 = 2^0 + 2^3 + 2^7$. Moreover, this additive unity is clearly tied to the Combinatorial Hierarchy \cite{Bastin}, based on the Catalan-Mersenne series starting with 3, because $N_4 = 10 = 3 + 7 and 3+7 + 127 = 137 = N_{16} + 1$. The following term $N_{32}  = 528$ cannot be compared with the enormous Lucas-Mersenne Large Number $2^{127}- 1$, so it is the last term of the Hierarchy. 

This Lucas number appears in the precise (ppb) formula of table 3, in liaison with 137. Moreover $32^2 - N_{32} =  496$ is the above dimension of the superstring gauge group SO(32), and \textit {the third perfect number}, see below the importance of this fact, not reckognized untill now.


\section{The Planck law connection with the Bernouilli function}

 Moreover, $a$ is tied in the superstring 9D space with the two constants of the Planck law, whose kernel is the Bernouilli fonction, $x/(1-e^{-x})$, \textit {central in the Atiyah's work} [2]. These are the reduced Wien displacement constant $w$, and the number of photons $16\pi \zeta(3)$ in a volume $\lambda^3$, with $\lambda = hc/kT $, corresponding to one photon by volume $l_{ph}^3$:

\begin{equation}
(16\pi\zeta(3))^3/w^4 = \lambda^5l_{Wien}^4/l_{ph}^9 \approx \pi_a^3a    \Rightarrow     \pi_a: 3;7;16;17p_an/p
\end{equation}

As in the preceeding case, this is a symbolic rationalisation. This is the single formula obtained by computer in this article.

\section{Conclusion}

This work proves the existence of an Ultimate Massive String Theory, and, according to the Atiyah's testimony [3], that the octonion algebra and the sporadic groups are related, opening a new field in mathematical research. In the scope of a computational Cosmos, the main parameters appears as calculation bases, and he Wyler's theory is completely rehabilitated  The most imminent prediction is that the James Webb telescope will show old galaxies in the far field, instead of the predicted so-called ``dark age''.

\section {Aknowledgements}

The authors thanks Anatole Kelif and Eva Tokgoz for many discussions, including with Atiyah, and Denis Gayral for technical assistance


\bibliographystyle{unsrt}  
%\bibliography{references}  %%% Remove comment to use the external .bib file (using bibtex).
%%% and comment out the ``thebibliography'' section.
%%% Comment out this section when you \bibliography{references} is enabled.
\begin{thebibliography}{99}
\bibitem{Hirzebruch} Hirzebruch F. Topological methods in algebraic geometry. Springer 1966.\\
\bibitem{Bott} M. Atiyah, R. Bott, V. Patodi, ``On the heat equation and the index theorem'' Invent. Math. , 19 (1973) pp. 279--330.\\
\bibitem{Singer} M. Atiyah, I. Singer, ``The index of elliptic operators IV'' Ann. of Math. , 93 (1971) pp. 119--138. \\
\bibitem{Alvarez} L. Alvarez-Gaume, ``Supersymmetry and the Atiyah Singer index theorem'' Comm. Math. Phys. , 90 (1983) pp. 161--170.\\
\bibitem{Atiyah} Atiyah M. https://hitsmediaweb.h-its.org/Mediasite/Play/35600dda1dec419cb4e99f706197a3951d. \\ 
\bibitem{Sanchez} F.M. Sanchez, V. Kotov, M. Grosmann, D. Weigel, R. Veysseyre, C. Bizouard, N. Flawisky, D. Gayral, L. Gueroult, Back to Cosmos.\\
\bibitem{Shannon} Shannon C.E. ``A Mathematical Theory of Communication'' Reprinted with corrections from The Bell System Technical Journal, Vol. 27, p. 379–423, 623–656, July, October, 1948.)\\
\bibitem{Atiyah1} Atiyah M. Private Communication (december 2018).\\
\bibitem{Tanabashi} Tanabashi M. et al. (Particle Data Group), Phys. Rev. D98, 030001 (2018), and 2019 update.\\
\bibitem{Wyler} Wyler A., ``L'espace symetrique du groupe des equations de Maxwell'' C. R. Acad. Sc. Paris, t. 269, 743-745 (1969). Wyler A., C.R. Acad. Sci, Paris ``Les groupes des potentiels de Coulomb et de Yukawa''. C. R. Acad. Sc. Paris, t. 272, 186-188 (1971).\\
\bibitem{Apostol} Apostol T. Modular functions and Dirichlet Series in Number Theory. Springler-Verlag. New-York (1990).\\
\bibitem{Lovelace} Lovelace C. (1971) Pomeron form factors and dual Regee cuts, Physics Letters B34 (6) 500-506.\\
\bibitem{Stark} Stark H.M. A complete determination of the complex quadratic fields of class-number one, Michigan Math. J., vol. 14,‎ 1967, p. 1-27  \\
\bibitem{Conway} Conway, John Horton; Norton, Simon P. (1979). ``Monstrous Moonshine''. Bull. London Math. Soc. 11 (3): 308--339.\\
\bibitem{Borcherds} Borcherds, Richard (1992), ``Monstrous Moonshine and Monstrous Lie Superalgebras'', Invent. Math., 109: 405--444.\\ 
\bibitem{Green} Green, M. Schwarz J. (1984)  ``Anomaly cancellations in supersymmetric D = 10 gauge theory and superstring theory''. Physics Letters B. 149: 117.\\
\bibitem{shray} Shray J. (1994) Octonions and Supersymmetry, PhD thesis.  http://ir.library.oregonstate.edu/xmlui/handle/1957/35649. \\
\bibitem{Koide} Koide Y., Fermion-Boson Two-Body Model of Quarks and Leptons and Cabibbo Mixing  Lett. Nuovo Cimento 34, 201 (1982). v 
\bibitem{Eddington} Eddington A, Fundamental Theory, Cambridge.\\
\bibitem{Bastin} Bastin T. and Kilmister C.W., Combinatorial Physics (World Scientific, 1995).\\
\bibitem{Sanchez1}  Sanchez F.M., Holic Principle, Entelechies, ANPA 16, Sept. 1995. Bowden K.G., 324--343.\\
\bibitem{Hooft} Hooft 't Th Holographic Principle. ArXiv: https://arxiv.org/pdf/hep-th/0003004.pdf. \\
\bibitem{Bousso} Bousso R., The Holographic Principle, Review of Modern Physics, vol 74, p.834 (2002).\\
\bibitem{Friedman} Friedman W. et al, The Carnegie-Chicago Hubble Program. VIII. An Independent Determination of the Hubble Constant Based on the Tip of the Red Giant Branch, arxiv : 1907.05922.\\ 
\bibitem{Sanchez2} Sanchez F.M., Kotov V. and Bizouard C., ``Towards a synthesis of two cosmologies: the steady- state flickering Universe''. Journal of Cosmology, vol 17. (2011).\\
\bibitem{Quinn} Quinn T, Speake C, Parks H, Davis R. 2014 The BIPM measurements of the Newtonian constant of gravitation, G. Phil.Trans. R. Soc. A372: 20140032. http://dx.doi.org/10.1098/rsta.2014.0032. \\
\bibitem{Sternheimer} Sternheimer J., Musique des particules elementaires, CRAS, 297, II, 829--834 (1983).\\
\bibitem{Weigel} Veysseyre R., Veysseyre H., and Weigel D. Counting, and Symbols of Cristallographic Point Symmetry Operations of Space En. AAECC 5, 53--70 (1994).\\

\end{thebibliography}
\end{document}
\endinput
%%
%% End of file `elsarticle-template-num.tex'.

\documentclass[a4paper,9pt]{article}
\usepackage{conference}
\usepackage{latexsym}
%\usepackage[utf8]{inputenc}
\usepackage[utf8x]{inputenc}
\usepackage{textcomp}
\usepackage[english]{babel}
\usepackage{amssymb,amsfonts,amsmath}
\usepackage{graphicx}
\usepackage{pstricks}
\usepackage{cite}
\usepackage{hyperref}
\usepackage[varg]{txfonts}

\usepackage{booktabs}       % professional-quality tables

%\usepackage[letterpaper, landscape, lmargin=0.25in, rmargin=1.25in]{geometry}
\usepackage{tabularx}
\usepackage{enumerate}
%\usepackage{enumitem}
\usepackage{multicol}
\pagenumbering{gobble}
\title{Hubble Table}
%\author{author}
\date{01/20/2020}
%\title{Hubble Table \emph{conference} preprint}
\title{Atiyah's Physics-Mathematics Unification confirms Permanent Oscillatory Cosmology}
\author{
  F.M. Sanchez
  Department of Physics\\
  Paris 11 University\\
  Paris, FRANCE \\
  \texttt{hol137@yahoo.fr} \\
  %% examples of more authors
   \And
 M.H. Grosmann \\
  Department of Photonics\\
  University of Strasbourg\\
  Strasbourg, FRANCE \\
  \texttt{michel.grosmann@me.com} \\
   \And
 D. Weigel \\
  Department of Crystallography\\
  CNRS\\
  Paris, FRANCE \\
  \texttt{dominiqueweigel118@gmail.com} \\
   \And
 R. Veysseyre \\
  Department of Crystallography\\
  Ecole Centrale\\
  Paris, FRANCE \\
  \texttt{r.veysseyre@gmail.com} \\
  %% Coauthor \\
  %% Affiliation \\
  %% Address \\
  %% \texttt{email} \\
}


\begin{document}
\maketitle

\begin{abstract}
%\lipsum[1]
The Atiyah's testimony about the Physics-Mathematics unification is confirmed by 71 formula
giving the Hubble radius, with 7 correlating to $10^{-9}$. This confirms the Permanent Oscillatory
Cosmology. The identification with the Eddington statistical formula gives $G$, compatible with the
$10^{-5}$ precise BIPM measurement and the $10^{-6}$ precise quasar non-Doppler oscillation. The $\pi$ rationalization process in a computing Cosmos validates the Wyler's theory. 

\end{abstract}


% keywords can be removed
\keywords{Quantum Physics \and Number theory \and Cosmos \and Holography \and Crystallography}



\label{sec:headings}

\section{Introduction}



  
    From Hirzebruch's work \cite{Hirzebruch}, which revolutionized geometry and topology, Sir Michael Atiyah, Raoul Bott \cite{Bott} and Isadore Singer \cite{Singer} introduced the  index theory, acclaimed by theoretical physicists \cite{Alvarez}. Following this path, on the advice of the physicist Gerard t'Hooft, Atiyah looked for the determination of the electrical constant $a \approx 137.0359991$ \cite{Atiyah}.
    
    
    At the 2018 Heidelbergh Laureate Forum, he showed that the extrapolation of the Euler formula  $e^{2\pi} = 1$ to the quaternions leads to the 'Atiyah constant' $\Gamma = \gamma a/\pi $. Meanwhile, he rehabilitated the Eddington bare electrical constant, the prime number 137, and announced that the resolution of the Riemann conjecture appears as a "bonus". Moreover, the four forces would be connected to the four principal algebra, whose the octonion non-associative one would be tied to the gravitation constant $G$ in a future work \cite{Atiyah}. 
       

\section{The scope}
   The precise measurement of $G$ is very difficult, as well as the Hubble radius $R$. While the Atiyah's work does not seem to give the $a$ value, nor the Riemann conjecture solution, we show in this article how this Atiyah constant $\Gamma$ ties $R$ and $G$, to $10^{-9}$ (ppb) precision, by using the number 137, through the electro-weak Fermi constant.
   
   
     Moreover, $\Gamma$ enters the Topological Axis \cite{Sanchez} \ (Fig. 1), both in connection with the Higgs boson and the galaxy group radius, a crucial cosmic distance. Now, the Topological Axis shows the Bott periodicity, typical of octonion algebra. According to Atiyah, the later would be connected with the sporadic groups, opening a new field in mathematics \cite{Atiyah1}.
     
     
     The present article confirms that the totality of the 26 sporadic groups are involved, as well as the multi-dimensional crystallography and the periodic atomic table. This confirms the \textit {arithmetic} approach of Eddington, Hirzebruch and Atiyah.

\section{The cosmic liaison between $a$ and the weak mixing angle}

Thus, the physical parameters would be mathematical constants of an unknown arithmetical domain. So, their « fine-tuning» is not due to hazard in a Chaotic Multiverse, but are of mathematical origin in a single Cosmos. The main result of preceding study is that the Cosmos volume, with length unit the Hydrogen radius, involves $a^a$, showing that \textit{$a$ is an optimal computation base}  \cite{Sanchez}:

\begin{equation}
    (4\pi /3) (R_C/r_H)^3 \approx a^a/\pi \approx (1/ln2)^{\sqrt{pH}} \approx (13/3)^{p/4} \approx (1/sin^2\theta)^{n/4} 
\end{equation}



where $p$, $H$ and $n$ are the proton-electron, hydrogen-electron and neutron-electron mass ratios, and 13/3 the fraction associated to the decomposition 16 = 13 + 3. This corresponds to the value $sin^2\theta)\approx 0.231235$, compatible with the measured weak mixing angle 0.21322(4) \cite{Tanabashi}
  
\section{The Rehabilitation of Wyler\'s theory}
The presence of an excess $\pi$ in the above formula suggests that $\pi$ is also a computation base for the cosmos.This is indeed the case in the even Riemann series.


Atiyah did not consider this computation point of view, but insisted on the analogy of his procedure with that of Archimedes for calculing $\pi$ \cite{Atiyah}. But, \textit {in the hypothesis that the cosmos is a computer}, the cosmos cannot use the mathematical Archimedes constant $\pi$, which is an idealisation, \textit {otherwise any time calculation would be infinite}. Its decomposition is an unresolved problem, but the first terms are : $3, 7, 16, -293.634$, where the fourth term, hightly singular, is so close (3 ppm) to $1 + n/2\pi$, where $n$ is the neutron/electron mass ratio.

\begin{equation}
\pi : 3, 7, 16, -(1+n/2\pi)
\end{equation}


In the famous Wyler formula \cite{Wyler} 

\begin{equation}
(3\sqrt(a/4))^8 = 120 \times \pi_W^{11}
\end{equation}


implicitely tied to the 11D supergravity space, the development of  $\pi_W$ shows an analogy with the above, apart the insertion of \textit {the singular prime 163} :

\begin{equation}
\pi_W : 3, 7, 16,- 163/2, -(1+n/4\pi) ~~~~\Rightarrow ~~~~    a \approx 137.03599908399
\end{equation}


This number 163 is the last of the Heegner-Stark numbers \cite{Stark}, directly connected to the famous Ramanujan quasi-whole number.


Moreover, according to Atiyah \cite{Atiyah1}, an approximation of $\pi$ appears directly in $(a^2-137^2)^{1/2} = \pi_{a,137} : 3, 7, 10, a_s$, , where the forth term is very close to the inverse strong coupling constant $1/a_s \approx 0.1179(10)$ \cite{Tanabashi}. The proton-electron mass ratio of Wyler \cite{Wyler} is the simple formula 

\begin{equation}
 p_W = 6\pi^5    
\end{equation}

which is \textit {the product of the area of a cube of side $\pi$ with its volume}. Taking the above value $\pi_W$, this gives $p_{W, a, 137} \approx 1834$, which is of central pertinence in cosmology: indeed it connects both with $R$ and $R_1$, the one-electron universe radius \cite{Sanchez} (table 1).

\section{The Central Role of the Modular Number 744}

The \textit{Ramanujan quasi-whole number} $N_R = exp(\pi\sqrt(163))$, is tied to the Dedekind eta function, which plays a role in bosonic string theory \cite{Apostol}\cite{Lovelace}, wholly rehabilitated by the Topological Axis (Fig.1).

This large number is also tied to the modular function $j$, whose Fourier series shows linear combinations of dimensions of the irreductive representations of the Monster group. With $q = 2\pi x$ :

\begin{equation}
j(x) = 1/q + 744 + 196884 q + ...
\end{equation}

In particular 196884 = D + 1 where D = 196883 is the Monster group order. This was called the Monster Moonshine \cite{Conway}. It was shown that string physics makes a bridge between two separated mthematical domains \cite{Borcherds}. 


But the q-independent number number 744 is unexplained. It is related to the above fraction 13/3 by the relation involving the Monster and the bosonic string dimension 26:

\begin{equation}
  O_M^{1/26^2} \approx 744^{1/6^2}\approx (1+1/d_e)^{1/2}   
\end{equation}

where $d_e \approx 1.001159$ is the electron magnetic moment excess.

One  remarks it is 744 = (3/2)496, where 496 is the dimension of the superstring gauge group SO32, a necessary conditions for a superstring theory to make sense \cite{Green}.


But 10-dimensional string theory is the version of the theory that uses octonions algebra \cite{Schlay}. So, it seems that the Atiyah's conjecture was correctly prophetic. The present study confirms this connection octonion-sporadics from cosmology.

\section{The Decisive Role of the term $a^a$ }

Moreover, the product of the 6 pariah sporadic groups is directly tied with $a^a$ and $F/a$ (table 7), $F$ being the ratio Fermi/electron. 

Indeed, the above canonical number $a^a$  is also very close to the Lucas-Lehmer term $S_9 = g_3^{2^9}$, where $ g_3 = 2 + \sqrt(3)$ is the generator of quasi-whole numbers. Now, the Lucas-Mersenne Large Number $2^{127} - 1$ plays a central role in Coherent Cosmology \cite{Sanchez}. It is prime because it is a divisor of the number $S_{125}$, which appears to connect also in cosmology (first formula of table 2).

Now, $a^a$ connects also directly with the famous Ramanujan quasi-whole number $N_R = exp(\pi \sqrt(163))$, tied to the above Heegner-Stark number 163, which shows dramatic correlations:



\begin{equation}
lnR_N = \pi \sqrt(163)  \approx lna ln\tau  \approx  lnp ln\mu
\end{equation}

\begin{equation}
N_R^{\tau/\mu}  \approx a^a
\end{equation}

\begin{equation}
g(1)  \approx \tau/\mu  \approx  2a_s \approx  a^s(N_{ph}^{1/3}/n_{ph})             \Rightarrow              1/a_s \approx  0.11892
\end{equation}

where $g(k) = exp(2^k)/k$ is the Topological Function (Fig 1), while $p$, $\mu$ and $\tau$ are respectively the masses of Proton, Muon, and Tau by respect to the Electron one.  Now they seem to be related to Topological Axis Function $g(1)$ and the strong coupling $a_s$, while  $N_{ph}$ and $n_{ph}$  are the number of photons in the Cosmos and the critical sphere of radius $R$, \textit{the Hubble radius, proved to be invariant in the present work}.


\section {The Koide-Wyler ppb relation}

While $\mu$ is measured to 0.1 ppm, $\tau$ is badly measured. The Koide relation \cite{Koide}, always unexplained, has shown correct predictability for the $\tau$ mass, proving \textit{the present standard particle theory is badly insufficient}. This relation writes in the most symmetrical way connecting with the above Wyler formula, in the following \textit{ppb formula, which confirms the specified  value}: $\mu = (Fa/\sqrt{pH})^{1/2}$\cite{Sanchez}.


\begin{equation}
(1 + \mu + \tau)/2 = (1 + \sqrt{\mu} + \sqrt{\tau})^2/3 = p_K \approx 6\pi_K^5 (1+(\mu / \tau)^2) ~~~~\pi_K: 3,7,16,-(2\times 137)^{2/3}
\end{equation}

Indeed, the fermions Mu and Tau are complete mystery in the standard model. However, Eddington predicted the tau fermion, 35 years before its surprising discovery, calling it « heavy Mesotron », with a right order prediction of its mass \cite{Eddington}.

\section{The rehabilitation of 137 from the electroweak coupling constant}

 This was very surprising because the Eddington theory, accused of pythagorism, was rejected. But he also predicted the importance of the $N_16 = 136$ elements of a symmetrical matrix 16 x 16, giving 137 by adding unity, whose pertinence is confirmed by the very precisely (0.1 ppm) measured electroweak coupling (inverse) factor 

\begin{equation}
a_w = (2\times137 \Gamma)^3
\end{equation}
 
Atiyah presented this number by the form $137 = 2^0 + 2^3 + 2^7$. Moreover, this additive unity is clearly tied to the Combinatorial Hierarchy \cite{Bastin}, based on the Catalan-Mersenne series starting with 3, because $N_4 = 10 = 3 + 7 and 3+7 + 127 = 137 = N_{16} + 1$. The following term $N_{32}  = 528$ cannot be compared with the enormous Lucas-Mersenne Large Number $2^{127}- 1$, so it is the last term of the Hierarchy. 

This Lucas number appears in the precise (ppb) formula of table 3, in liaison with 137. Moreover $32^2 - N_{32} =  496$ is the above dimension of the superstring gauge group SO(32), and \textit {the third perfect number}, see below the importance of this fact, not reckognized untill now.




\section{The Planck law connection with the Bernouilli function}

 Moreover, $a$ is tied in the superstring 9D space with the two constants of the Planck law, whose kernel is the Bernouilli fonction, $x/(1-e^{-x})$, \textit {central in the Atiyah's work} [2]. These are the reduced Wien displacement constant $w$, and the number of photons $16\pi \zeta(3)$ in a volume $\lambda^3$, with $\lambda = hc/kT $:

\begin{equation}
(16\pi\zeta(3))^3/w^4 = \lambda^5l_{Wien}^4/l_{ph}^9 \approx \pi_a^3a    \Rightarrow     \pi_a: 3;7;16;17p_an/p
\end{equation}

As in the preceeding case, this is a symbolic rationalisation. This is the single formula obtained by computer in this article.


\section{The Holographic Fine-Tuning with the Universal Critical Radius}

Among the 30 or so free parameter of the present standard model, the Nature seems to favor some ones (Hierarchy Principle \cite{Sanchez}). They distinguish themselves as being measured with high precision, so the Table 7 does not include the quarks, neither the neutrinos. The tables 4 to 8 resume the involved quantities and their notations. 

Moreover the Nature seems to be ruled by the Holographic Principle and its Diophantine form, the Holic Principle, presented in 1994 at ANPA (Cambridge)  \cite{Sanchez1}. Orsay University gave a sabbatical year (1997-1998) to FMS, in order to develop the application of the Holographic and Holic principle in theoretical physics. In the three first minutes of this sabbatical year, FMS found, by the most elementary method, based on the universal constants, half the length 13.80(2) Gyr (billion light-year), deposed in a closed letter in March 1998 at the French Academy. 


So, to show that the Hubble radius is constant, it was sufficient, in elementary dimensional analysis, to replace the speed $c$ by the mean masses of the 3 main particles in Atomic Physics. Note that the general use of $c$ = 1 seems to have precluded this discovery before. Also, for most theorists, the proton is not a so fundamental particle as the electron. But this is reductionist point of view. In fact, the proton mass is fairly well measured (table 7), while the quark masses are not, as recalled above.
 
    This 2 factor is typical of the critical Schwarzschild radius $2 = Rc^2/GM$, and is also presented in the Archimedes testimony, as the ratio between the perimeter and the area of a disk with radius unity, the first historic holographic relation, \textit {as expressively noticed by Atiyah }. So, the critical radus is given by an holographc relation defining a space quantum $l_0$:
    
    \begin{equation}
        \pi (R/l_P)^2 = 2\pi R/l_0
    \end{equation}
    
    So this universal radius $R$ may be considered as the radius for which, in an homogeneous Universe (the basic cosmological principle), the included mass reaches the above critical value\cite{Sanchez}. Thus each space quantum (topon) in the cosmos is the center of a sphere with universal radius $R$. Indeed, the above critical relation  
    
    
    This permits to resolve the question of the enormity of the vacuum energy by pushing down the Plank wall by a factor $10^61$, resolving also the vacuum quantum energy dilemna  \cite{Sanchez} . 
    
    At the same epoch, some theorists, as t' Hooft \cite{Hooft}, introduced also the Holographic Principle, but could not apply it to the Universe, believing the Hubble radius is variable.
    
    
    In fact these authors applied the above disk area to a blackhole, calling it 'Bekenstein-Hawking entropy'\cite{Bekenstein}, but, instead of considering the real disk, they considered the sphere area, introducing the factor 1/4. 
    
    
    In fact it was shown that, starting from the real disk, a 3D sphere surface can be generated by rotating it around a diameter, leading, via an universal quantification tying the electron to the Lucas Number and the proton to the Eddington Number \cite{Sanchez}. \textit {This explains why the cosmos is so large}. Indeed, it tries to mimic a continuous space, to use approximations of $\pi$ in  the calculation.
    
    The critical factor 2 can be also considered as the ratio between the areas of a unit-radius sphere to the circonference of diametral disk. The extension to the 3D volume gives the nominal Cosmic Microwave Background (CMB) wavelength, corresponding to 2.73 Kelvin, in function of atomic and molecylar hydrogen wavelengths:\cite{Sanchez}:
    
    \begin{equation}
        2\pi R/lambdabar_e \approx 4\pi (\lambdabar_H)^2 \approx (4\pi/3) (\lambdabar_{CMB}/lambdabar_{H2}^3
    \end{equation}

    The series of 69 formula presented in this article confirm the invariance of both the Hubble radius and the cosmos temperature, as well as the background (cosmos) temperature. Let us recall that the Hubble radius is defined by $R = c/H_0$, where $H_0 = v/d$ is the Hubble constant, which implies the apparent speed $v$ in the red-shift of a $d$-distant galaxy$ v = c \Delta \lambda/\lambda$. So, there is \textit {the direct simpler relation} $\Delta \lambda/\lambda =  d/R$. 
    

    
    The so-called standard Universe age is 13.80(2) billion years \cite{Tanabashi}, while the Hubble radius deduced from the super novae 1a is $R_{SN1a} \approx 13,6(6)$ Gly \cite{Freeman}\textit{This article shows this cannot be related to any age}, since this length is given by a series of 68 formula implying invariant quantities, including the cosmic background. 
    
    
    This recalls the 14 formula presented by Jean Perrin in 1909 to prove definitely the real existence of atoms. Here, the task is to show the existence of an ultimate theory of massive strings in a dramatic re-interpretation of standard cosmology:\textit{ the Big Bang becoming permanent, and the Multiverse becoming coherent: each point is the center of a $R$-radius sphere}. This means that the Universe is destroyed and reconstructed in an high-frequency oscillation. This permits to consider matter as an matter-antimatter oscillation \cite{Sanchez2}. 
    
    This opens to the possibility that \textit {dark matter, whose existence is proven} by the connection with the Eddington large number $N_{Edd}$, (table 1), would be a quadrature oscillation.
    
    \section{The connection with Diophantine and Eddington physics}

These formula give a $R$ value compatible with the following Diophantine analysis. The movement $(r,v)$ of a mobile in a gravitational central field has the form $r v^2 = Gm_G$, where $m_G$ is a characteristic mass. Viewing the third Kepler law as a Diophantine one, it resolves in  $T^2 = L^3 = n^6$, thus $L = n^2$,the orbital law in the Hydrogen atom, characterized by $rv = \hbar/m_{P})$. So, there is a kind of symmetry between $G$ and $\hbar$. Consider the following system, using the two principal masses, the electron and proton's ones: 

\begin{equation}
  r v^2 = Gm_e
  \end{equation}
  \begin{equation}
r v = \hbar/m_p  
\end{equation}

Thus, with the Planck mass $m_P = (\hbar c/G)^{1/2}$ : 
\begin{equation}
c/v = m_P^2/m_em_p = \sqrt(M/m_e)~~~~~~~~   M = m_P^4/m_em_p^2
\end{equation}

By identifying this mass with the critical mass of the Universe, this is the statistical solution \cite{Durham} of the Large Number Question by Eddington  : $R = 2 \sigma \sqrt{(M/m_0)}$, where the reference mass $m_0$ is identified to $m_e$ and the standard deviation $\sigma$ to $\hbar/cm_pm_H$, in conformity with the gravitational Hydrogen molecule model \cite{Sanchez}. The optimized value for $G$ follows:
\begin{equation}
R = 2\hbar^2/Gm_em_pm_H  ~~~~  \Rightarrow G \approx 6.67545375 \times 10^{-11}  kg^{-1}m^{3}s^{-2}    
\end{equation}

which is compatible with the BIPM value \cite{Quinn}, precise to 10 ppm, but not with the standard value \cite{Tannabashi} which is the incongruous mean between discordant measurements. Moreover, this $G$ value is compatible with the value corresponding to the elimination of c between the gravitational and electroweak coupling constants (among the last formula of Table 2), leading to specify the non-Doppler quasar Kotov period $t_K \approx 9600.591457$ sec.



\section{The connection with the Periodic Table of Elements}

     In fact the pythagorism is in accordance with a quantum computation world ruled by Arithmetics. In particular the four smaller dimension numbers of the Topological Axis (Fig. 1) : 2, 6, 10, 14 identify with the atomic numbers of the Periodic Table spectroscopic series : $s, p, d, f$ . The Periodic Table contains 19 such series, corresponding to 118 atoms : 7s + 6p + 6d + 2f = 118 (atomic number of the Oganesson nucleus). 

     Now the periods are distinct from the principal quantum number, so that the periods starting from the second one are double. So, the above number of atoms decompose in $118/2 = 59 = 1 + 3s + 3p + 2d + f$. Now these numbers 19 and 59 are the Crystalline Ponctual Symmetry Operation numbers ($PSO_{Cr}$), respectively negative and positive in 6D crystallography [10] (Table 9) : $k_{6-} = 19$, $k_{6+} = 59$. Note that this dimension d = 6 corresponds to k = 1 in the Topological Axis. By separating the last series f + 1 = 15, the theoretical decomposition 137 = 107 + 30 is justified by the sum $137 =  k_{6-} + 2k_{6+} = 7(s +1) + 6(p +1) + 4(d +1) + 2(f +1)$. Note that $s + 1 = 3$ and $p + 1 = 7$ are the first terms of the above Combinatorial Hierarchy\cite{Bastin} .

     Consider all the series in the Topological Axis, by introducing the supplementary series $g, h, i, j$ of dimensions 18 ; 22 ; 26 ; 30, corresponding to the higher part of the Topological Axis, after the 16 which is the central dimension, this leads to
     
     \begin{equation}
      8s + 7p + 6 d + 5f + 4g + 3h + 2i + j = 408 = 3 \times 136   
     \end{equation}
      
     
     This writes, in function of the 10 D point symmetry operation numbers :  $k_{10-} = 165$ and $k_{10+} = 419$: $SO3 \times 136 = 419 - 11 = 165 + 35$, and $419-165 = 2 \times 127 =  35 + 11$. Note that the later is the supergravity dimension number and that $128/3^5$ is the classical musical limma. 
     
     
     But the superstring theory is only coherent in 9D space. For every odd dimension number, $k_{(2n - 1)-} = k_{(2n - 1)+} = k_{2k-}$ so the above combination type $k_- + 2k_+$ is for 9D: $3 \times 165 = 495$, the canonical reduced number attached to the above perfect number 496. This number 495 is associated to the Higgs boson (Fig. 1) and to the smallest sporadic group, the Mathieu one, of order 16×495. Note that the couple 495-496 has the same Euler index (240) and the same Carmichael-lambda index (60). This could be unique, defining 496 as a super-perfect number. Note also that 496 is close to the 20th root of the Monster order.
     
 \section{connection with the high-dimensional crystallography}
    
    Note that the above number $419$ is the number of positive Point Operation in 10D cristallography, while $417$ is the number of trivial ones \cite{Weigel} (table 7).
    
    One notes the the above decomposition of $137 = 2\times 59 + 19$  traduces in terms of crytalline PSO as:
    
    \begin{equation}
    K_{5+}+K_{6+}+K_{7+} = K_{5+} + K_{6+} +K_{7+} = 137   
    \end{equation}
    
    while:
    
    \begin{equation}
    K_{10+} +K_{11+}+K_{12+} = K_{11-} +K_{12-}+K_{13-} = 1839   
    \end{equation}
    
    this number 1839 is the whole number closest to the neutron-electron mass ratio.
    
    The sum of the mean values $K_{d} = (K_{d+} + K_{d-})/2$ untill the dimension $d = 12$ is 1836, which is the entire part of the proton-electron mass ratio. Note that this sum limited to $d = 7$ gives 138.
    
     So it seems that the dimension $12$ will play a role in the future string theory.
    
    Note that the roots of the crystallographic algebraic equation of degree $n$ are of type $exp(i2\pi m/l)$, where $l$ and $m$ are whole numbers such that $ l \leq n $ and $ 1 \leq m \leq l $ ,  : this is similar to the above spectroscopic series. Such an unexpected connection needs also further study.   





\section{The connection with the topological function}

Using the Holographic Principle, the cosmic quantities associated to this critical radius R are defined in the table 5, in particular the Cosmos radius, which shows a dramatic connection with the topological term $g(7)$ :

\begin{equation}
R_{C}/\lambdabar_{e} g(7) \approx  \lambdabar_{e}/6l_P \approx F^5/6a^3 \approx (\lambdabar_{CMB}/r_H)^3 \approx  (am_p/m_e)^4  
\end{equation}

This induces the discovery of the Central Gravito-Electroweak relation :

\begin{equation*}
F^5/a^3 \approx \eta P    
\end{equation*}{}
  

with  $F = (2^137 G)^{3/2}$, the Fermi-Atiyah-Sanchez, factor, specifying the measured value of  $F$ (Table 7) with the help of the above Atiyah's constant $G = a \gamma/ \pi$, where appears the Veysseyre-Weigel-Sanchez factor $\eta = 419/417$, very close to the Sternheimer limma $2^{1/144}$ \cite{Sternheimer}]. Note that 419 is the number of positive Point Operation in 10D cristallography. while 417 is the number of trivial ones \cite{Weigel} (table 9).


Moreover, this confirms that the Cosmos is the real source of the background radiation [5]:
\begin{equation}
F^5  \approx 6(\lambdabar_{CMB}/\lambdabar_{e})^3 \Rightarrow  T_{CMB}  \approx  2.725820 K  (mes : T_{CMB}  \approx  2.7255(6) K)) 
\end{equation}


The graviton mass, calculated from the double step holo-tachyonic propagation is associated with that of the photon. This graviton mass connects directly with $g(6)$ :

\begin{equation}
m_N/m_{gr} \approx g(6)/(1+1/\mu)^2 \Rightarrow    t_K  \approx  9600.65 sec ~~(mes : t_K \approx 9600.60(1) sec)    
\end{equation}

implying the mass ratio Muon-Electron. 

The important fourth formula of Table 2 confirms that they are only three families of particles. Indeed this number 3 enters the definition of the energy density of the neutrino background (Table 5). 


\begin{table*}
  \hskip-1.0cm\begin{tabular}{llllllllllllll}
    \toprule
    \multicolumn{14}{c}{Table 7. Cristallographic $PSO_{Cr}$}                  \\
    \cmidrule(r){1-14}
    %\ $E^d$ & E & $E^1$  & $E^2$ & $E^3$ & $E^4$ & $E^5$ & $E^6$ & $E^7$ & $E^8$ & $E^9$ & $E^{10}$ & $E^{11}$ & $E^{12}$ \\
    \midrule
    %$K_{d+}$  & 1 & 1 & 5 & 5 & 19 & 19 & 59 & 59 & 165 & 165 & 419 & 419 & 1001 \\
    
     %$K_{d-}$  &  & 1 & 1 & 5 & 5 & 19 & 19 & 59 & 59 & 165 & 165& 419 & 419 \\
     
    % $2K_{d+} + K_{d-}$  & 2 & 3 & 11 & 15 & 43 & 57 & 137 & 177 & 389 & 495 & 1003 & 1257& 2421 \\
     
      %$(K_{d+} - K_{d-})/2 + d$  &  &  & 4 & 3 & 11 & 5 & 26 & 7 & 61 & 9 & 137 & 11& 303 \\
      
      \ $E^{(d)}$ & $E^{(0)}$ & $E^{(1)} $ & $E^{(2)}$ & $E^{(3)} $& $E^{(4)}$ &$ E^{(5)}$ &$ E^{(6)} $&$ E^{(7)}$ &$ E^{(8)}$ & $E^{(9)}$ &$ E^{(10)} $&$ E^{(11)} $&$ E^{(12)}$ \\
    \midrule
    $K_{d+}$  & 1 & 1 & 5 & 5 & 19 & 19 & 59 & 59 & 165 & 165 & 419 & 419 & 1001 \\
    
     $K_{d-}$  &  & 1 & 1 & 5 & 5 & 19 & 19 & 59 & 59 & 165 & 165& 419 & 419 \\
     
      $K_{d} = (K_{d+} + K_{d-})/2$  & & 1 & 3 & 5 & 12 & 19 & 39 & 59 & 112 & 165 & 292 & 419 & 710 \\
      
      $\Sigma K_{d}$ &  & 1 & 4 & 9 & 21 & 40 & 79 & 138 & 250 & 415 & 707 & 1126 & 1836 \\
      
      $K_{(d-1)+} +K_{d+}+ K_{(d+1)+}$  &  & 7 & 11 & 29 & 43 & 97 & 137 & 283 & 389 & 749 & 1003 & 1839 & - \\

      

    \bottomrule
  \end{tabular}
  \label{tab:table}
\end{table*}

Note that the roots of the crystallographic algebraic equation of degree $n$ are of type $exp(i2\pi m/l)$, where $l$ and $m$ are whole numbers such that $l \leq n$ and $l \leq m \leq l$,  : this is similar to the above spectroscopic series. Such an unexpected connection needs also further study.




    %{ Eddington predicted the tau fermion, 35 years before its surprising discovery, calling it «heavy Mesotron», with a right order prediction of its mass [13]. This was very surprising because the Eddington theory, accused of pythagorism, was rejected. But he also predicted the importance of the $N{_16} = 136 elements of a symmetrical matrix $16\times 16$, giving 137 by adding unity, whose pertinence is confirmed by the above Central relation $a_w = (2×137 G)^3$. Atiyah presented this number by the form $137 = 2^0 + 2^3 + 2^7$.} 

     %{Moreover, this additive unity is clearly tied to the Combinatorial Hierarchy [14], based on the Catalan-Mersenne series starting with 3, because $N_4 = 10 = 3 + 7 and 3+7 +127 = 137 = N_{16} + 1$. The following term N_{32}  = 528 cannot be compared with the enormous $2^{127 =- 1$, so this is the last term of the Hierarchy which is also \textit(the Lucas Number, the largest prime number demonstrated without computer). This number appears in the ppb precise formula of table 3, in liaison with 137. Moreover $32^2 - N_{32} =  496$ is the dimension of the superstring gauge group SO(32), and the third perfect number, see below the importance of this fact, apparently not reckognized untill now.}

    


\section{Conclusion}

This work proves the existence of an Ultimate Massive String Theory, and, according to the Atiyah's testimony [3], that the octonion algebra and the sporadic groups are related, opening a new field in mathematical research. In the scope of a computational Cosmos, the main parameters appears as calculation bases, and he Wyler's theory is completely rehabilitated  The most imminent prediction is that the James Webb telescope will show old galaxies in the far field, instead of the predicted so-called « dark age».



\bibliographystyle{unsrt}  
%\bibliography{references}  %%% Remove comment to use the external .bib file (using bibtex).
%%% and comment out the ``thebibliography'' section.
%%% Comment out this section when you \bibliography{references} is enabled.
\begin{thebibliography}{99}
\bibitem{Hirzebruch} Hirzebruch F. Topological methods in algebraic geometry. Springer 1966.\\
\bibitem{Bott} M. Atiyah, R. Bott, V. Patodi, "On the heat equation and the index theorem" Invent. Math. , 19 (1973) pp. 279--330.\\
\bibitem{Singer} M. Atiyah, I. Singer, "The index of elliptic operators IV" Ann. of Math. , 93 (1971) pp. 119--138. \\
\bibitem{Alvarez} L. Alvarez-Gaume, "Supersymmetry and the Atiyah Singer index theorem" Comm. Math. Phys. , 90 (1983) pp. 161--170.\\
\bibitem{Atiyah} Atiyah M. https://hitsmediaweb.h-its.org/Mediasite/Play/35600dda1dec419cb4e99f706197a3951d. \\ 
\bibitem{Sanchez} F.M. Sanchez, V. Kotov, M. Grosmann, D. Weigel, R. Veysseyre, C. Bizouard, N. Flawisky, D. Gayral, L. Gueroult, Back to Cosmos.\\
\bibitem{Atiyah1} Atiyah M. Private Communication (december 2018).\\
\bibitem{Tanabashi} Tanabashi M. et al. (Particle Data Group), Phys. Rev. D98, 030001 (2018), and 2019 update.\\
\bibitem{Wyler} Wyler A., "L'espace symetrique du groupe des equations de Maxwell" C. R. Acad. Sc. Paris, t. 269, 743-745 (1969). Wyler A., C.R. Acad. Sci, Paris "Les groupes des potentiels de Coulomb et de Yukawa". C. R. Acad. Sc. Paris, t. 272, 186-188 (1971).\\
\bibitem{Apostol} Apostol T. Modular functions and Dirichlet Series in Number Theory. Springler-Verlag. New-York (1990).\\
\bibitem{Lovelace} Lovelace C. (1971) Pomeron form factors and dual Regee cuts, Physics Letters B34 (6) 500-506.\\
\bibitem{Conway} Conway, John Horton; Norton, Simon P. (1979). "Monstrous Moonshine". Bull. London Math. Soc. 11 (3): 308--339.\\
\bibitem{Borcherds} Borcherds, Richard (1992), "Monstrous Moonshine and Monstrous Lie Superalgebras", Invent. Math., 109: 405--444.\\ 
\bibitem{Green} Green, M. Schwarz J. (1984)  Anomaly cancellations in supersymmetric D = 10 gauge theory and superstring theory". Physics Letters B. 149: 117.\\
\bibitem{shray} Shray J. (1994) Octonions and Supersymmetry, PhD thesis.  http://ir.library.oregonstate.edu/xmlui/handle/1957/35649. \\
\bibitem{Koide} Koide Y., Fermion-Boson Two-Body Model of Quarks and Leptons and Cabibbo Mixing  Lett. Nuovo Cimento 34, 201 (1982). v 
\bibitem{Eddington} Eddington A, Fundamental Theory, Cambridge.\\
\bibitem{Bastin} Bastin T. and Kilmister C.W., Combinatorial Physics (World Scientific, 1995).\\
\bibitem{Sanchez1}  Sanchez F.M., Holic Principle, Entelechies, ANPA 16, Sept. 1995. Bowden K.G., 324--343.\\
\bibitem{Hooft} Hooft 't Th Holographic Principle. ArXiv: hep-th/003004 (2000). \\
\bibitem{Bousso} Bousso R., The Holographic Principle, Review of Modern Physics, vol 74, p.834 (2002).\\
\bibitem{Friedman} Friedman W. et al, The Carnegie-Chicago Hubble Program. VIII. An Independent Determination of the Hubble Constant Based on the Tip of the Red Giant Branch, arxiv : 1907.05922.\\ 
\bibitem{Sanchez2} Sanchez F.M., Kotov V. and Bizouard C., 'Towards a synthesis of two cosmologies: the steady- state flickering Universe'. Journal of Cosmology, vol 17. (2011).\\
\bibitem{Quinn} Quinn T, Speake C, Parks H, Davis R. 2014 The BIPM measurements of the Newtonian constant of gravitation, G. Phil.Trans. R. Soc. A372: 20140032. http://dx.doi.org/10.1098/rsta.2014.0032. \\
\bibitem{Sternheimer} Sternheimer J., Musique des particules elementaires, CRAS, 297, II, 829--834 (1983).\\
\bibitem{Weigel} Veysseyre R., Veysseyre H., and Weigel D. Counting, and Symbols of Cristallographic Point Symmetry Operations of Space En. AAECC 5, 53--70 (1994).\\

\end{thebibliography}





\subsection{Tables}


\begin{itemize}
\item [Table1] See Table section \ref{tab1:table1} 40 formulas for the Hubble radius with 1\% precision.
\item [Table2] See Table section \ref{tab2:table2} 22 formulas for Hubble radius with $2.10^{-4}$ precision.
\item [A3] See Table section \ref{tab:table3} 7 ppb precise formulas for $R \approx 13.8119768$ Gly.
\item [A4] See Table section \ref{tab:table4} physical constants.
\item [A5] See Table section \ref{tab:table5} cosmic constants.
\item [A5] See Table section \ref{tab:table5} Adimensional cosmic constants.
\end{itemize}




 
\subsection{Table1}
\begin{table*}
  \hskip-2.0cm\begin{tabular}{llll}
    \toprule
    \multicolumn{4}{c}{Table 1. 40 formulas for the Hubble radius, with better precision than 1 \%}                   \\
    \cmidrule(r){1-4}
   \#     & Formula     & Value~~(Gyr) & Remarks \\
    \midrule
    
    
    1 & $(20/3)N_{Edd}Gm_H/c^2$ & 13.79 & Confirms Eddington Large number and black matter existence [3] \\
    2 & $2\hbar^2/Gm_em_pm_n$ & 13.80 & obtained in a 3 minutes calculation (1997) by dimensional analysis withput c\\
    3 & $2\hbar^2/Gm_em_p^2$ & 13.82 & theoretical radius of a mono-atomic star\\
    4 & $\lambdabar_{e} g(6)$ & 13.82 & with the topological function $g(k)=exp(2^{k+1/2})/k$ for k=6 (d=26, critical dimension)\\
    5 & $\lambdabar_{e} (|tau/\mu)^{32}/w$ & 13.83 & $6g(6) = g(1)^{32}$\\     
    6 & $(2\lambdabar_{e}/3)(\lambdabar_{CMB}/\lambdabar_{H2})^3$ & 13.90 & 3D holographic term in $2\pi R/\lambdabar_{e} \approx 4\pi (\lambdabar_{p}/l_P)^2 \approx (4\pi /3) (\lambdabar_{CMB}/\lambdabar_{H2})^3$ \\
    7 & $\lambdabar_{e} S_4^5$ & 13.80 & holographic 5D extension\\
    8 & $\lambdabar_{e} \Gamma^{55/2}$ & 13.80 & implies $s_4 \approx \Gamma^{11/2}$\\
    9 & $\lambdabar_{e} exp(j\sqrt (137/a) - \Gamma)$ & 13.82 &confirms the Atiyah and Sternheimer constants\\ 
    10 & $\lambdabar_{e} exp((p^2-p_{W,a,137}^2 - j/\pi)$ & 13.81 & with $p_{W,a,137} = 6(a^2 - 137^2)^{5/2} \approx 1833.99827$~~ confirms Wyler's theory\\ 
    11 & $\lambdabar_{e} exp(\sqrt (p^2-p_{W,a,137}^2)/d_e^2)$ & 13.81 & with $p_{W,a,137} = 6(a^2 - 137^2)^{5/2} \approx 1833.99827$ ~~ confirms Wyler's theory\\ 
    12 & $\lambdabar_{p} {(WZ)}^{4}$ & 13.80 & specifies the Carr and Rees relation $a_G \approx W^8$ [5] \\
    13 &  $(2l_K^3/r_e)^{1/2}$ & 13.75 & from holographic relation $\pi(R/l_K)^2 approx 2\pi l_K/r_e$  \\
    14 &  $l_K(3(r/l_P)^2)^{1/3}$ & 13.69 & from holographic relation $(4\pi/3) (R/l_K)^^3  approx 4\pi (l_K/r_e)^2$ \\
    15 &  $(R_{C}r_e^2)^{2/3}/l_k$ & 13.70 & from $\sqrt(3) l_K^3  \approx R_{C}r_el_P$ \\
    16 & $\lambdabar_{e} ^{11/3}/l_P^2 \lambdabar_{CMB}^{2/3}$ & 13.87 & confirms the thermal photon background\\
    17 & $2\lambdabar_{CNB}^6/\lambdabar_e ^3 \lambdabar_{CMB}^2$ & 13.83 & confirms the statistical neutrino background\\
    18 & $2\lambdabar_{e} a_s^2 W^7$ & 13.86 & confirms the Holic Principle \\
    19 & $2\lambdabar_{e} (FZ)^{7/2}$ & 13.95 & confirms the Holic Principle \\
    20 & $\lambdabar_{e} 2^{128}$ & 13.90 & $R/2 \approx 2^{127}$ Lucas Large Number, last term of the Combinatorial Herarchy\\
    21 & $\lambdabar_{e} \pi^{155/2}$ & 13.80 & $\pi$ as a calculation basis (Riemann series): $2^{1/155} \approx \pi^{1/256} \approx (2\pi)^{1/(3\times 137)}$ \\
    22 & $4P^3\lambdabar_{e} l_{WCMB} /R_N$ & 13.82 & from the Holo-thermal holographic relation : $e^a \approx 4\pi (R_N/l_{WCMB} )^2 \approx (2\pi /3) (r_p/l_P)^3$  \\
    23 & $(2\pi^{32}P\lambdabar_{e})^2 /R_N$ & 13.80 & ties to $l_{WCMB}/l_P \approx \pi^{64}$\\        
    24 & $R_N a^a/\Pi_{hap} (R_{C}/l_P)^3/\Pi_{26}$ & 13.81 & ties the Grandcosmos hologram radius to the 20 happy family sporadic groups\\  
    25 & $R_N (R_{C}/l_P)^3/\Pi_{26}$ & 13.79 & ties the Grandcosmos to the 26 sporadic groups\\   
    26 & $\lambdabar_{F} P^3 /p^7$ & 13.80 & P and p computation bases\\      
    27 & $\lambdabar_{F} P^2 e/8$ & 13.81 &  related to $\sqrt a  \approx 32/e$ \\     
    28 &  $\lambda_{e} O_M^{7/10}$ & 13.94 &  related to $O_M^{7/10} \approx 496$, dimension of the superstring SO32 gauge group  \\
    29 & $(\lambdabar_{Ryd} n^{4})^2/\lambdabar_p$ & 13.81 & tied to $ct_K/\lambdabar_e \approx aFWZn$ \\ 
    30 & $(\lambda_{CMB}/(j+1))^2/l_P$ & 13.80 & yieds to the central cosmo-biologic relation [5]: $\sqrt(Rl_P) \approx \lambda_{mam}$ \\
    
    
    31 & $(\lambda_{CMB}^4/j\sqrt{E_3})^{1/2}/l_P$ & 13.84 & implies $j/a \approx \sqrt{ln2} \approx 1/\zeta(3)$\\ 
   
    32 & $(\lambdabar_{e} (2R/R_N)^{210})$ & 13.85 & Confirms the Holic Principle and the  Grandcosmos hologram with radius $R_N$  \\
    
    33 & $R_N(R_N \pi^{1/3}/O_M\lambdabar_{e})^{1/127}$ & 13.77 & Confirms the Monster  \\
    34& $(\lambdabar_{e} (\tau /p)^{140}/2$ & 13.77 & confirms the Eddington's proton-tau symmetry \\
    35& $R_N (O_M O_B/n_{ph})^2$ & 13.77 & confirms the large spradic groups. $(O_M O_B/2)^{-1/a} \approx sin^2\theta \approx ln^42$ \\
    36 & $R_N (\pi O_M O_B/3)^2 / exp(e^6)$ & 13.90 & confirms the pertinence of $e^6 \approx \pi^4 + \pi^5  \approx sin^2\theta \approx ln^42$ \\
    37 & $(\sqrt{3}/2)\lambdabar_{e}g_3^{8a_s}$ & 13.84 & Confirms the Lucas-Lehmer series $g_ 3^{2^n}$\\
    38 & $2\lambdabar_{e} N_R^{1+\sqrt{137}}/(R_N/l_P)^3$ & 13.86 & Confirms the Ramanujan Number pertinence\\
    39 & $R_{C} (e^\gamma/R_N^7)^{1/2}$ & 13.81 & Confirms the Superspeed ratio $C/c = R_{GC}/R$\\
    40 & $\lambda_{e} \sqrt(a) \times 744^{\sqrt(163)}$   & 13.78 & Confirms  liaison Modular-sporadic $O_B \approx 744^{ \sqrt(137)}$ \\ 
      
    \bottomrule
  \end{tabular}
  \label{tab1:table1}
\end{table*}

\subsection{Table2}
\begin{table*}
  \hskip-2.0cm\begin{tabular}{llll}
    \toprule
    \multicolumn{4}{c}{Table 2. 22 formulas for Hubble radius, with better precision than $2 \times 10^{-4}$}                   \\
    \cmidrule(r){1-4}
   \#     & Formula     & Value~~(Gly) & Remarks \\
    \midrule    
    
    
   1 & $(l_P^2 \lambdabar_{e})a_s^2 N_L (\lambdabar_{CMB}/r_H)^6$ & 13.810 & confirms the cosmic role of the strong coupling $a_s$ \\
 
   2 & $\lambdabar_{e} ((a-136)E_3^{\sqrt a})^{1/2}$ & 13.814 & $E_3 = e^{e^e} \approx E_4^{1/ap} \approx e^{3e+7}\approx \tau \times 8a a\approx e^7/8$ \\
   3 & $\lambdabar_{e} \Pi_{26}^{1/9}/(j+e)$ & 13.813 & with the product of the 26 sporadic group orders\\
   4 & $(\Pi_{26}^2(\lambdabar_e/j)^{18}R_N/2)^{1/19}$ & 13.813 & $j^{18} \approx a^{17} lna$\\
   5 & $\lambdabar_{e} a^{5a/38}$ & 13.812 & a computation basis\\
   6 & $\lambdabar_{e} (D/3 - a)^8$ & 13.813 & empiric $D/3 -a -1 \approx 2\mu p_{hol}a^{-1/2}$\\
   7 & $(\lambdabar_{e}^2/R_N) (137/(16 \times 136)) g_3^a$ & 13.815 & confirms the Lucas-Lehmer generator $g_3 ; g_3 +1/g_3 = 4$\\
   8 & $R_1 a_s a^3 N_L e^{-2}P^{-2}$ & 13.813 & by comparison with $Gm/c^2$\\
   9 & $R_1 (8/\sqrt{3a})^{1/7}$ & 13.8118 & from relations between photon numbers \\
   10 & $\lambdabar_{e} ((e^{4e-1/a} - ln^2(P^4/a^3))/2)^{1/2}$ & 13.8117 & from the geo-dimensional couple Universe-Grandcosmos\\
   11 & $\lambdabar_{F} e P^2E_2^4(pn)^{-1/2}$ & 13.8126 &tied to $H/8 \approx E_2^2 = e^{2e}$\\
   12 & $(\lambdabar_{e}^2/l_P) (j/16)^{16}E_2^2 d_e \sqrt 2$ & 13.8120 & liaison j-matrix $16 \times 16$\\
   13 & $3^{1/137} R_{GC}^{2/3} r_e^{4/3} /l_K$ & 13.8124 & confirms the liaison Grandcosmos-quasar period\\
   14 & $O_M^{d_e pH\sqrt\beta / 24D}$ & 13.8115 & confirms the monster and its dimension D\\
   15 & $(a 137^{-1/2}(4\pi F)^{-2} \lambdabar_{e}^4 l_{ph}^3 (\lambdabar_{CMB})/l_P^8)^{1/7}$ & 13.81189 & comes from $\sqrt{2n_{ph}/n_n} \approx (u^U)/(u_{CMB}+u_{CNB})$\\
   16 & $R_N exp(-2/e^2)$ & 13.81195 & empiric\\ 
   17 & $2\beta \lambdabar_{e} j^{17} (4\pi)^2 \sqrt{137}$ & 13.81198 & j calculation basis \\
   18 & $\lambda_{e} (3j^j/2H)^{1/6}$ & 13.81199 & j and a : related computation bases : $(j^j)^{5/4} \approx a^a$\\
   19 & $\beta F P^{3/2} (n/p)^{7/2} 2 \pi$ & 13.81198 & proton-neutron symmetry  \\ 
   20 & $(45\lambdabar_{CMB}^7/4(p+5)/\lambda_{CNB}^3)^{1/2}/l_P$ & 13.81197 & confirms $T_{CMB} and p+5 \approx n^2/p \approx H^5/p^4$\\
   21 & $4l_Kp^4 \sqrt{p/H}/\beta d_e$ & 13.81198 & confirms the non-Doppler quasar period \\
   22 & $2(l_K/F)^2/\lambdabar_{e}$ & 13.81198(3) & from elimination of $c$ between gravitational and electroweak couplings \\
  
  
    \bottomrule
  \end{tabular}
  \label{tab2:table2}
\end{table*}


\begin{table*}
  \hskip-2.0cm\begin{tabular}{llll}
    \toprule
    \multicolumn{3}{c}{Table 3. 7 ppb precise formula for $R \approx 13.8119768$ Gly}                   \\
    \cmidrule(r){1-3}
    \#     & Formula  & Remarks \\
    \midrule
    
    
   
    7 & $2\lambdabar_{e} (pn/H^{2})(g(5)/\ln(2-1/ja^2))^2$   & confirms the Topological axis $g(5)^2/g(6) = 25/6 \rightarrow \ln(2) \approx 2\sqrt(3/5)$  \\
    
    6 & $xR_1^2/R_N with x = (11/4)^{1/610}$ &  confirms the statistical term 11/4 ; $2/x^{137} \approx \ln(11/4) \approx d_e^{10}$ \\
    
    5 & $(20/3)N_{Edd} exp((4 \pi_0)^{-3})/\lambdabar_n$ & $\pi_0 =  (22a - 377/2)/(7a - 60) \leftrightarrow \pi_{Arch} = 22/7  \pi_{Ptol} = 377/120 = 2 + 137/120$  \\
    
     4 & $\lambdabar_{e} g(6)/(1+\sqrt(137^2+\sqrt(136))/jn)$  & Confirms $137=136+1$ \\
    3 & $(\lambdabar_{e} 2^{128})(1-(137^2+\pi^2+e^2)/pH)$ & shows a symmetry between $\pi$, e and 137, prolongating $ a \approx (137^2 + \pi^2)^{1/2}$ \\
     2 & $(\lambdabar_{e} 2^{137})(\gamma^2 n^6 / 137^2 \Gamma^{11})$ & superstring liaison 11D-9D, with $\Gamma$, the Atiyah constant \\
    1 & $(\lambdabar_{e} 2^{128}/d_e^2(m_H/m_p)^6$  & empiric [5], separates the neutron from $\Gamma \gamma^2 d_e^2 \approx (p\Gamma^2 \sqrt(137)/2 \sqrt(2) Hn)^6 \approx a_s$ \\
    
    \bottomrule
  \end{tabular}
  \label{tab:table3}
\end{table*}

\begin{table*}
  \hskip-2.0cm\begin{tabular}{lllll}
    \toprule
    \multicolumn{5}{c}{Table 4. physical constants}                  \\
    \cmidrule(r){1-5}
    \ name & Symbol  & unit  & Value & imp (ppb) \\
    \midrule
  
 Relativity speed     & c   & $m s^{-1}$   & 299792428 & exact \\
 Planck constant     & h   & J s   & $6.62607015 \times 10^{-34}$ & exact \\
 Reduced Planck constant $h/2\pi$    & $\hbar$   & J s   & $1.05457181 \times 10^{-34}$ & "exact" \\
 Official Gravitation constant   & $G_{off}$ & $kg^{-1} m^3 s^{-1}$ & $6.67430 \times 10^{-11}$  &  contested\\
 Optimized Gravitation constant   & G & $kg^{-1} m^3 s^{-1}$  & $6.67545375\times 10^{-11}$  &  ppb\\
 Fermi constant  & $G_F$ & $J m^3$   & $61.435851 \times 10^{-62}$  &  500\\
 Electron mass $m_e = m_p/p = m_H/H = m_n/n$  & $m_e$ & kg  & $9.1093837015 \times 10^{-31}$  &  0.3\\
 Mean mass $(m_e  m_p =  m_n )^{1/3}$ & m & kg  & $9.1093837015\times 10^{-28}$  &  0.3\\
 Boltzman pseudo constant (unity convertor) & $k_B$ & $J /K$  & $1.380649 \times 10^{-23}$  &  exact \\
 Wien displacement constant  $\lambda_Wien \times T = hc/k_Bw$ & $k_B$ &  m K  & $2.897771995 \times 10^{-3}$  &  "exact"\\
 Electron reduced wavelength $\hbar/m_ec$ & $\lambdabar_e$ &  m   & $3.861592675\times 10^{-13}$  & 0.3\\
 Electron classical radius $\hbar/am_ec$ & $r_e$ &  m   & $2.817940322\times 10^{-15}$  & 0.45\\
 Hydrogen Bohr radius $a(1+1/p)\lambdabar_e$ & $r_H$ &  m   & $5.294654092 \times 10^{-15}$  & 0.45\\
 Proton radius  & $r_p$ &  m   & $8.8\times 10^{-16}$  & contested\\
 Planck length $(\hbar G /c^3)^{1/2}$ & $l_P$  & m  & $1.61639471 \times 10^{-35}$ & this work ppb  \\
 Rydbergh correction constant $(H-p)^{-1}$ & $\beta$  & -  & 1.000026597 &   \\
 Planck ratio $m_P/m_e$ & P  & -  & $2.389015907 \times 10^{22}$ & this work ppb  \\
 Gravitational coupling constant $R/2\lambdabar_e = p^2/pH$ & $a_G$   & -  & $1.691936467 \times 10^{38}$ & this work ppb  \\
 Electroweak coupling constant $F^2 = (2\gamma\times 137)^3$ & $a_w$   & -  & $3.283374406 \times 10^{11}$ & this work ppb  \\
  
    
    \bottomrule
  \end{tabular}
  \label{tab:table4}
\end{table*}






\begin{table*}
  \hskip-2.0cm\begin{tabular}{lllll}
    \toprule
    \multicolumn{5}{c}{Table 5. cosmic constants}                   \\ 
      \cmidrule(r){1-5}
     name & Symbol   & unit   & Value & imp (ppb) \\
 \midrule
   
    
    Official Hubble-Lemaitre so-called "present" constant & $c/H_0$ & Gly & 13.80(2)    & $1.5 \times 10^6$ \\
  
    Critical Universal radius $2\hbar^2/Gm_em_pm_H=2GM/c^2=2a_G\lambdabar_{e}$ & R &  Gly & 13.81197677  & this work ppb\\
   
   Universal mass $Rc^2/2G = m_P^4/m_em_pm_H$ & M & kg & $8.796524777 \times 10^{52}$ & this work ppb \\ 
   
   Universal energy density & $u_U$ & $J/m^3$ & $8.459065716 \times 10^{-10}$ & this work ppb \\
   
   Grandcosmos hologram Nambu radius & $R_N$ &  m & $1.712894163 \times 10^{26}$ & this work ppb\\
   
   Grandcosmos radius & $R_{GC}$ &  m & $9.075773376 \times 10^{86}$ & this work ppb \\
   
   Universal mono-electron radius $\lambdabar_e exp((\pi^2/6-1)a(1+1/p)+1-\gamma)\approx g_3^{a/2}/4$ & $R_1$ &  m & $1.492365473 \times 10^{26}$ & this work ppb \\
    
     Official CMB temperature & $T_{CMBoff}$ & K & $2.7255(6)$ & $2 \times 10^5$ \\
    
   Grandcosmos (CMB) temperature & $T_{CMB}$ & K & $2.725820138$ & this work ppb \\
   
 Neutrino temperature  $(CNB)T_{CMB}/ (4/11)^{1/3}$ & $T_{CNB}$ & K & $1.945597343$ & this work ppb \\
 
 CMB energy density $(\pi^{2/15})\hbar c/ \lambdabar_{CMB}^4 \approx (2a_s^2)^2 u_U$ & $u_{CMB}$ & $J/m^3$ & $4.176762758 \times 10^{-14}$ & this work ppb\\
 
 CMB photon density $16 \pi \zeta (3)/\lambda_{CMB}^3$  & $l_{ph}^{-3}$  & $m^{-3}$   & $410.871743 \times 10^6 m^{-3}$ & this work ppb\\
 
  CNB energy density $u_{CMB} = (3\times (7/8) \times (4/11)^{4/3})$ & $u_{CNB}$ & $J/m^3$ & $2.84572016\times 10^{-14}$ & this work ppb\\
  
  Non-Doppler quasar period & $t_{Kmes}$ & sec & 9600.60(1) & 1000 \\
  
 Optimized Non-Doppler quasar period $\lambdabar_e (a_Ga_w)^{1/2}/c$ & $t_{K}$ & sec & 9600.591457 & this work ppb \\
 
 Equivalent number of neutrons in the critical sphere & $n_n$ & - & $5.251883912 \times 10^{79}$ & this work ppb \\
 
 Number of photons in the critical sphere  & $n_{ph}$ & - & $3.840045866 \times 10^{87}$ & this work ppb \\
 
 Number of photons in the Grandcosmos  & $N_{ph}$ & - & exp(621.949984) & this work ppb \\
 
 Equivalent number of Hydrogen atoms in the Grandcosmos  & $N_H$ & - & exp(603.8432382) & this work ppb \\
   
   

 \bottomrule
  \end{tabular}
  \label{tab:table}
\end{table*}


   

\begin{table*}
  \hskip-2.5cm\begin{tabular}{llll}
    \toprule
    \multicolumn{4}{c}{Table 6: adimensional primary constants}                   \\
    \cmidrule(r){1-4}
    name & symbol    & value & imp (ppb) \\
    \midrule
    
    Euler-Napier constant  & e    & 2.718281828459042 & 'exact' \\
    
    Archimedes constant & $\pi$    & 3.1459265358979 & 'exact' \\ 
    
    Euler-Mascheroni constant & $\gamma$    & 0.57721566490153 & 'exact' \\
    
    
    Apery constant & $\zeta(3)$    & 1.202056903159594 & 'exact' \\
    
    Lucas-Lehmer generator $\sqrt3 + \sqrt4 g_3+1/g_3 = 4$  & $\zeta(3)$    & 3.73205080756888 & 'exact' \\
    
    Wien constant $w = 5(1-e^{-w})= hc/k_BT\lambda_{Wien}$  & w    & 4.961142317443 & 'exact' \\
    
    Eddington Electric constant ~~~~$e^\pi \approx a/ln(ea)   \approx a-j$  & a    & 137.035999084 & 0.15 \\
    
    Electron magnetic moment ~~/~~ Bohr magneton  & $d_e$    & 1.00115965218128 & 0.15 \\
    
    Atiyah constant & $\Gamma$    & 25.17809724196  & 0.15 \\ 
     Eddington Large Number & $N_{Edd}$    & $136 \times 2^{256}$  & exact \\
     
     Lucas Large Prime Number & $N_L$    & $2^{127}-1$  & exact \\
     
     Monster group order & $O_M$    & $2^{46}\cdot 3^{20} \cdot 5^9 \cdot 7^6 \cdot 11^2 \cdot 13^3 \cdot 17\cdot 19 \cdot 23 \cdot29 \cdot 31 \cdot 41 \cdot 47 \cdot 59 \cdot 71$  & exact \\
     
     Monster dimension & D    & $47 \cdot 59 \cdot 71 = 196883$   & exact \\
     
     Baby-Monster group order & $O_B$    & $2^{41}\cdot 3^{13} \cdot 5^6 \cdot 7^2 \cdot 11 \cdot 13 \cdot 17\cdot 19 \cdot 23 \cdot 31 \cdot 47$  & exact \\
     
     Happy Family order product & $\Pi_{hap}$   & exp(674.5210288)  & exact \\
     
      Pariah Family order product & $\Pi_{par}$   & exp(166.7658991)  & exact \\
      
      Measured Fermi/Electron $m_F/m_e$ & $F_{meas}$   & 573007.362  & 250 \\
      
      Fermi Atiyah Sanchez mass ratio: $(2\gamma \times 137)$ & F   & 573007.3652  & 0.22 \\
      
       Proton/Electron mass ratio $m_p/m_e$ & p   & 1836.15267343  & 0.06 \\
       
       Hydrogen/Electron mass ratio $H = p+1 -(p/a(p+1))^2/2$ & H  & 1837.15266014  & 0.06 \\
      
     Neutron/Electron mass ratio  & n & 1837.15266014  & 0.06 \\
     
     Measured Muon/Electron mass ratio  & $\mu_{meas}$ & 206.7682830  & 22 \\
     
     Sanchez Muon/Electron mass ratio $(Fa/\sqrt{pH})^{1/2}$  & $\mu$ & 203/7682869  & 0.1 \\
     
     Measured Tau/Electron mass ratio  & $\tau_{meas}$ & 3477.23  & $7\times 10^4$ \\
     
     Koide $\tau : (1+\mu+\tau)/2 = (1+\sqrt\mu+\sqrt\tau)^2/3$ & $\mu$ & 3477.441701  & 0.1 \\
     
     Measured W boson/Electron mass ratio  & $W_{meas}$ & 157297  & $1.5 \times 10^5$ \\
     
    Sanchez W boson/Electron mass ratio $137^2\Gamma/3d_e$  & W & 157340.1093  & 0.15 \\
     
    Measured Z boson/Electron mass ratio  & $Z_{meas}$ & 178450  & $2.3 \times 10^4$ \\
     
     Sanchez Z boson/Electron mass ratio $137^2\Gamma/3d_e$  & Z & 178451.7402  & 0.15 \\
     
    
     
      
     
   \bottomrule
  \end{tabular}
  \label{tab:table}
\end{table*}







\end{document}





\documentclass[a4paper,9pt]{article}
\usepackage{conference}
\usepackage{latexsym}
\usepackage[utf8]{inputenc}
\usepackage[english]{babel}
\usepackage{amssymb,amsfonts,amsmath}
\usepackage{graphicx}
\usepackage{pstricks}
\usepackage{cite}
\usepackage{hyperref}
\usepackage[varg]{txfonts}

\usepackage{booktabs}       % professional-quality tables

%\usepackage[letterpaper, landscape, lmargin=0.25in, rmargin=1.25in]{geometry}
\usepackage{tabularx}
\usepackage{enumerate}
%\usepackage{enumitem}
\usepackage{multicol}
\pagenumbering{gobble}
\title{Hubble Table}
%\author{author}
\date{01/20/2020}

\title{Hubble Table \emph{conference} preprint}
%{F.M. Sanchez,\ M. Grosmann,\ B. Kress,\ N. Flawisky,\ L. Gueroult, \textit{Cosmic Holography}}
\author{
  F.M. Sanchez~hol137\thanks{Use footnote for providing further
    information about author (webpage, alternative
    address)---\emph{not} for acknowledging funding agencies.} \\
  Department of Physics\\
  Paris 11 University\\
  Paris, FRANCE \\
  \texttt{hol137@yahoo.fr} \\
  %% examples of more authors
   \And
 M.H. Grosmann \\
  Department of Photonics\\
  University of Strasbourg\\
  Strasbourg, FRANCE \\
  \texttt{michel.grosmann@me.com} \\
   \And
 D. Weigel \\
  Department of Cristallography\\
  CNRS\\
  Paris, FRANCE \\
  \texttt{d.weigel@me.com} \\
   \And
 R. Veysseyre \\
  Department of Cristallography\\
  Ecole Centrale\\
  Paris, FRANCE \\
  \texttt{r.veysseyre@me.com} \\
  %% Coauthor \\
  %% Affiliation \\
  %% Address \\
  %% \texttt{email} \\
}


\begin{document}
\maketitle

\begin{abstract}
%\lipsum[1]
Hubble Table
\end{abstract}


% keywords can be removed
\keywords{Quantum \and Holography \and Cosmos}


\section{Introduction}
Atiyah's Physics-Mathematics Unification confirms Permanent Oscillatory Cosmology

The Atiyah's testimony about the Physics-Mathematics unification is confirmed by 57 formula
giving the Hubble radius, with 6 correlating to $10^-9$. This confirms the Permanent Oscillatory
Cosmology. The identification with the Eddington statistical formula gives G, compatible with the
$10^-5$ precise BIPM measurement and the 10 -6 precise quasar non-Doppler oscillation.
From Hirzebruch's work [1], which revolutionized geometry and topology, Sir Michael Atiyah,
Raoul Bott and Isadore Singer introduced the index theory, acclaimed by theoretical physicists.
Following this path, on the advice of the physicist Gerard t'Hooft, Atiyah looked for the
determination of the electrical constant $a \approx 137.0359991$. 


\section{The cyclical ”Big-Bang” model}
\label{sec:headings}

At the 2018 Heidelbergh Laureate Forum,
he showed that the extrapolation of the Euler formula $exp(2\pi i) = 1$ to the quaternions leads to the
'Atiyah constant' $\Gamma = \gamma a/\pi$. Meanwhile, he rehabilitated the Eddington bare electrical constant, the
prime number 137, and announced that the resolution of the Riemann conjecture appears as a
"bonus". Moreover, the four forces would be connected to the four principal algebra, whose the
octonion non-associative one would be tied to the gravitation constant G in a future work [2].
The precise measurement of G is very difficult, as well as the Hubble radius R. While the Atiyah's
work does not seem to gives the a value, nor the Riemann conjecture solution, we show in this
article how this Atiyah constant $\Gamma$ ties R and G, to $10^{-9}$ precision, by using the number 137 and the
electroweak Fermi constant. Moreover, $\Gamma$ enters the Topological Axis (Fig. 1), both in connexion
with the Higgs boson and the galaxy group radius, a crucial cosmical distance. Now, the
Topological Axis shows the Bott periodicity, typical of octonion algebra. According to Atiyah, the
later would be connected with the sporadic groups, opening a new field in mathematics [3]. The
present article confirms that the totality of the 26 sporadic groups are involved, as well as the multi-
dimensional crystallography and the periodic atomic table. This confirms the arithmetic approach of
Eddington, Hirzebruch and Atiyah.
Thus, the physical parameters would be mathematical constants of an unknown arithmetical
domain. So, their « fine-tuning » is not due to hazard in a Multiverse, but are of mathematical origin
in a single possible Universe immersed in a Grandcosmos whose volume, with length unit the
Hydrogene radius, is $a^a /\pi$, showing that a is an optimal computation base [4]. Moreover, a is tied, in
the superstring 9D space with canonic constants $\omega$ (reduced Wien displacement constant) and
$16\pi \zeta(3)$ (the number of photons in a cube of side hc/kT) of the Planck law, whose kernel is the
Bernouilli function x/(1-e -x ), central in Atiyah's work.
$(16\pi \zeta(3))^3 /\omega^4 = \lambda 5 \lambda_{Wien}^4 /l_{ph}^9 \approx \pi a^3 \sqrt(a) \rightarrow \pi a : 3;7;16;17\pi a n/p$
This is the single formula obtained by computer in this article. It implies a slightly different value
$\pi^a$ . This is logical, because in the hypothesis of a computing cosmos, π must be rationalized,
otherwise the time calculation would be infinite. Recall that the arithmetic decomposition of $\pi$ : 3,
7, 15, 1, 15, 292.634 is a non-resolved problem of current mathematics. The proof of its pertinence
is that the dramatic sixth term is so close (3 ppm) to $n/2\pi$, where n is the mass ratio
neutron/electron.
Moreover, the product of the 6 pariah sporadic groups is directly tied with a a and F/a (table 7), F
being the ratio Fermi/electron. Among the 30 or so free parameters of the present standard model,
the Nature seems to favor some ones (Hierarchy Principle [4]). They distinguish themselves as
being measured with high precision, so the Table 7 does not include the quarks, neither the neutrinos. The tables 4 to 8 resume the involved quantities and their notations. Moreover the Nature
seems to be ruled by the Holographic Principle and its Diophantine form, the Holic Principle,
presented in 1994 at ANPA (Cambridge) [4]. In the three first minutes of his sabbatical year FMS
found, by the most elementary method, based on the universal constants, half the length 13.80(2)
Gyr (billion light-year), deposed in a closed letter in March 1998 at the French Academy.
This 2 factor is typical of the critical Schwarzschild radius $2 = Rc 2 /GM$, and is also presented in
the Archimedes testimony, as the ratio between the perimeter and the area of a disk with radius
unity, the first historic holographic relation, expressively noticed by Atiyah. This permits to resolve
the question of the enormity of the vacuum energy by pushing down the Plank wall by a factor 10 61
[4]. It was sufficient to replace the speed c by the product of masses of the 3 main particles in
Atomic Physics. At the same epoch, some theorists, as t'Hooft, introduced also the Holographic
Principle, but could not apply it to the Universe, believing the Hubble radius is variable.
The series of 57 formula presented in this article prove quite the contrary : the Hubble radius is
invariant, as well as the background temperature. Let us recall that the Hubble radius is defined by
$R = c/H_0$ , where $H_0 = v/d$ is the Hubble constant, which implies the apparent speed v in the redshift
of a d-distant galaxy $v = c \Lambda\lambda/\lambda$. So, there is the direct simpler relation $\Lambda\lambda/\lambda = d/R$. Moreover, this
radius may be considered as the radius for which, in an homogeneous Universe (the basic
cosmological principle), the included mass reaches the above critical value [4].

Voir Section \ref{sec:headings}.

\subsection{General Concept}
%\lipsum[5]
The so-called standard Universe age is $13.80(2) \times 10^9$ years, while the Hubble radius deduced
from the super novae 1a is $R_{(SN1a)} \approx 13,6(6)$ Gly ($10^9$ light-years). This article shows this cannot be
related to any age, since this length is given by a series of 57 formula implying invariant quantities,
including the cosmic background. This recalls the 14 formula presented by Jean Perrin in 1909 to
prove definitely the real existence of atoms. Here, the task is to show the existence of an ultimate
theory of massive strings in a dramatic re-interpretation of standard cosmology : the Big Bang
becoming permanent. This means that the Universe is destroyed and reconstructed in an high-
frequency oscillation. This permits to consider matter as an matter-antimatter oscillation [7]. This
leads to the possibility that dark matter, whose existence is proven by the connexion with the
Eddington large number $N_{Edd}$ , (tableau 1), would be a quadrature oscillation.

\subsubsection{New subsection}
%\lipsum[6]

Partisans et adversaires du modèle du Big Bang.

\section{New section}
\label{sec:headings}

Dans ce modèle il n'y a pas de « début » ni de « fin » de l'Univers.


\section{Conclusions}
\label{sec:headings}

Les récentes découvertes expérimentales en macro- et micro- Physique permettent d'imaginer un modèle d'Univers qui résoudrait élégamment les contradictions actuelles entre ces deux sous-disciplines. De nouvelles expériences (en préparation) devraient permettre de confirmer la pertinence de ce modèle face aux très nombreux autres actuellement en compétition. Nous essayons actuellement de le représenter sous forme d'un hologramme. Dans la conception de celui-ci nous avons réalisé des stéréoscopies dont l'une est présentée sur la Figure.

On y voit la représentation (étonnement sous forme d'une droite!) de divers paramètres tant de la MACRO- que de la micro- Physique.


\subsection{Figures}

See Figure \ref{fig:figure_label}. Pour plus d'explications. \footnote{http://vixra.org/abs/1904.0218}
 

\begin{table*}
  \hskip-2.0cm\begin{tabular}{llll}
    \toprule
    \multicolumn{4}{c}{14 formules au centi\`{e}me pr\`{e}s pour le rayon de Hubble: Bang permanent}                   \\
    \cmidrule(r){1-4}
   \#     & Formule     & Valeur(Gal) & Remarques \\
    \midrule
    1 & $2\hbar^2/Gm_em_pm_n$ & 13.80 & Calcul obtenu en 3mn (1997) par analyse dimensionelle sans c \\
    2 & $2\hbar^2/Gm_em_p^2$ & 13.82 & Rayon th\'{e}orique d'\'{e}toile monoatomique \\
    3 & $\lambdabar_e g(6)$ & 13.82 & Fonction topologique $g(k)=exp(2^{k+1/2})/k$ pour k=6 (d=26 valeur critique)\\
    4 & $\lambdabar_e S_4^5$ & 13.80 & $S_4=2\pi^2a^3$ aire de la sph\`{e}re 4D de rayon $a \approx 137.036$ \\  
    5 & $\lambdabar_e 2^{128}$ & 13.90 & $R/2 \approx 2^{127}$ Nombre de Lucas dernier terme de la hi\'{e}rarchie combinatoire \\ 
    6 & $\lambdabar_e \pi^{155/2}$ & 13.80 & $\pi$ base de calcul comme dans les s\'{e}ries de Riemann: $2^{1/155} \approx \pi^{1/256} \approx (2\pi)^{1/(3\times 137)}$ \\
    7 & $\lambdabar_p WZ^{4}$ & 13.80 & relation $a_G \approx W^8$ [3] \\
    8 & $\lambda_e O_M^{7/10}$ & 13.94 & $O_M^{7/10} \approx 496$, dimension du groupe de jauge supercorde SO32 \\
    9 & $(\lambdabar_{Ryd} n^{4})^2/\lambdabar_p$ & 13.81 & vient de $ct_K/\lambdabar_e \approx aFWZn$ \\ 
    10 & $(2\lambdabar_{e}/3)(\lambdabar_{CMB}/\lambdabar_{H2})^3$ & 13.90 & Conservation holographique \\
    11 & $(\lambda_{CMB}/(j+1))^2/l_P$ & 13.80 & vient de la relation centrale cosmo-biologique [3]: $\sqrt(Rl_P) \approx \lambda_{mam}$ \\
    12 & $(20/3)N_{Edd}Gm_H/c^2$ & 13.79 & Confirme Eddington et l'existence de la masse noire [3] \\
    13 & $(\lambdabar_{e} (2R/R_N)^{210})$ & 13.85 & Confirme le principe holique et l'hologramme du Grandcosmos de rayon $R_N$  \\
    14 & $2(ct_K/F)^2/\lambdabar_{e}$ & 13.81198(3) & Elimination de c entre les couplages gravitationel et interm\'{e}diaire [4] \\
    \bottomrule
  \end{tabular}
  \label{tab:table}
\end{table*}

\begin{table*}
  \hskip-2.0cm\begin{tabular}{llll}
    \toprule
    \multicolumn{4}{c}{5 formules au milliardi\`{e}me pr\`{e}s pour le rayon de Hubble: Bang permanent}                   \\
    \cmidrule(r){1-4}
   \#     & Formule     & Valeur(Gal) & Remarques \\
    \midrule
    1 & $(\lambdabar_{e} 2^{128})(1-(137^2+\pi^2+e^2)/pH)$ & 13.8119768  & Montre la sym\'{e}trie de $\pi$, e and 137 \\
    2 & $(\lambdabar_{e} 2^{128}/d_e^2(m_H/m_p)^6$ & 13.8119768  & Raccorde la constante d'Atiyah \\
    3 & $\lambdabar_e g(6)/(1+\sqrt(137^2+\sqrt(136))/jn)$ & 13.8119768 & Confirmation de 137=136+1 \\
    4 & $2\lambdabar_{e} (pn/H^{2})(g(5)/\ln(2-1/ja^2))^2$ & 13.8119767  & Implique la pertinence de $\ln(2) \approx 2\sqrt(3/5)$  \\    
    5 & $F_{AS}^5/a^3 \approx \eta \lambdabar_{e}/l_P$ & 13.8119767  & Relation Centrale  \\
    \bottomrule
  \end{tabular}
  \label{tab:table}
\end{table*}


\subsection{Lists}



\bibliographystyle{unsrt}  
%\bibliography{references}  %%% Remove comment to use the external .bib file (using bibtex).
%%% and comment out the ``thebibliography'' section.


%%% Comment out this section when you \bibliography{references} is enabled.
\begin{thebibliography}{1}

\bibitem{Sanchez3} Sanchez F.M. ``Towards the grand unified Holic Theory''. Current
Issues in Cosmology. Ed. J.-C. Pecker and J. Narlikar. Cambridge Univ. Press,
2006; p. 257--260.

\bibitem{Sanchez4} Sanchez F; M. ``Holic Principle: The coherence of the Universe`` (Sept 1995), Entelechies, 16th ANPA, 324--344.
\bibitem{Grosmann} Grosmann, M. and Meyrueis P. ``Optics and Photonics Applied to Communication and Processing''. SPIE.  Jan 1979.
\newblock Optics and Photonics Applied to Communication and Processing.
\newblock In {\em SPIE (SPIE), 1979 
  International Conference on}, pages . SPIE, 1979.
\bibitem{Grosmann2} Grosmann, M and Rebordão, José and Meyrueis, Patrick, 1985,02,p761--765,Propagation Of Waves In Optical Systems: Reformulation Of Huyghens Principle For Aspheric Systems,
volume 491, Proceedings of SPIE - The International Society for Optical Engineering, doi:10.1117/12.968010
\bibitem{Kress} Digital Diffractive Optics: An Introduction to Planar Diffractive Optics and Related Technology, by B. Kress, P. Meyrueis, pp. 396. ISBN 0-471-98447-7. Wiley-VCH , October 2000.
\newblock An Introduction to Planar Diffractive Optics and Related Technology.
\bibitem{Sanchez5} F.M. Sanchez, V. Kotov, M. Grosmann, D. Weigel, R. Veysseyre, C. Bizouard, N. Flawisky, D. Gayral, L. Gueroult, Back to Cosmos
\newblock {\em arXiv preprint viXra:1904.0218}, 2019.
\end{thebibliography}
\end{document}

%                                                                 aa.dem
% AA vers. 9.1, LaTeX class for Astronomy & Astrophysics
% demonstration file
%                                                       (c) EDP Sciences
%-----------------------------------------------------------------------
%
%\documentclass[referee]{aa} % for a referee version
%\documentclass[onecolumn]{aa} % for a paper on 1 column  
%\documentclass[longauth]{aa} % for the long lists of affiliations 
%\documentclass[letter]{aa} % for the letters 
%\documentclass[bibyear]{aa} % if the references are not structured 
%                              according to the author-year natbib style

%
\documentclass{aa}  

%
\usepackage{graphicx}
%%%%%%%%%%%%%%%%%%%%%%%%%%%%%%%%%%%%%%%%
\usepackage{txfonts}
%%%%%%%%%%%%%%%%%%%%%%%%%%%%%%%%%%%%%%%%
%\usepackage[options]{hyperref}
% To add links in your PDF file, use the package "hyperref"
% with options according to your LaTeX or PDFLaTeX drivers.
%

% F. M. Sanchez$^1\!$, \ M. Grosmann$^3$, \ D. Weigel$^4$, \ R. Veysseyre$^5$, \ L. Gueroult$^9$\\
% {\footnotesize  $^1$ Pr. Universit\'{e} Paris 11, Orsay, France (retired).\rule{0pt}{12pt}
% E-mail: hol137@yahoo.fr\\
% \\ $^2$Astronomer at Crimean Astrophysical Observatory CrAO, Nauchny, Crimea, Russian Federation. vkotov43@mail.ru
% $^3$ Pr. at Universit\'{e} Louis Pasteur, Strasbourg, France (retired). michelgrosmann@me.com
% $^4$ Pr. at Universit\'{e} of Rennes and Paris 6, France (retired). dominique.weigel@orange.fr
% \\ $^5$ Pr. at Ecole Centrale, Paris, France (retired). renee.veysseyre@gmail.com
% \\ $^6$Astronomer at OBSPM, Paris, France. christian.bizouard@obspm.fr
% \\ $^7$Architect / Math Instructor at ENSA Ecole Nationale Sup\'{e}rieure d'Architecture, Paris Malaquais, France. flawisky@free.fr
% \\ $^8$Computer Science Engineer. denis.gayral@epita.fr
% $^9$ Independant Researcher, France (retired). lgueroult@hotmail.com


%\address[1]{Pr. Universit\'{e} Paris 11, Orsay, France (retired)}
%\address[2]{Pr. at Universit\'{e} Louis Pasteur, Strasbourg, France (retired). michelgrosmann@me.com}
%\address[3]{Pr. at Ecole Centrale, Paris, France (retired). renee.veysseyre@gmail.com}
%\address[4]{Pr. at Universit\'{e} of Paris 6, France}
%\address[5]{Independant Researcher, former PhD instructor in Holography at ENS, Dept of Physics A2, Cachan, France}

\begin{document} 


   \title{Atiyah's Physics-Mathematics Unification confirms the Permanent Flickering Cosmology}

   \subtitle{I. Overviewing the Hubble Table}

   \author{F. M. Sanchez
          \inst{1}
          \and
          M. H. Grosmann\inst{2}\fnmsep\thanks{Universit\'{e} Louis Pasteur, Strasbourg, France}
          \and
          R. Veysseyre\inst{3}\fnmsep\thanks{Ecole Centrale, Paris, France}
          \and
          D. Weigel\inst{4}\fnmsep\thanks{Universit\'{e} of Paris 6, France}
          \and
          L. Gueroult\inst{5}\fnmsep\thanks{ENS, Cachan, France}
          }

   \institute{Universit\'{e} Paris 11, Physics Department,
              Orsay, France\\
              \email{hol137@yahoo.fr}
         \and
             Universit\'{e} Louis Pasteur, Photonics Department, Strasbourg, France, ...\\
             \email{michelgrosmann@me.com}
             \thanks{Retired}
         \and
             Ecole Centrale, Paris, France, ...\\
             \email{renee.veysseyre@gmail.com}
             \thanks{Agregee de mathematiques and prof honoraire at l'Ecole centrale of Paris}
         \and
             Universit\'{e} of Paris 6, Crystallography Department, France, ...\\
             \email{dominique.weigel@orange.fr}
             \thanks{Retired}
         \and
             ENS, Dept of Physics A2, Cachan, France, ...\\
             \email{lgueroult-at-hotmail.com}
             \thanks{Independant Researcher}                     
             }

   %\date{Received September 15, 2019; accepted March 16, 2020}

% \abstract{}{}{}{}{} 
% 5 {} token are mandatory
 
  \abstract
  % context heading (optional)
  % {} leave it empty if necessary  
   {The Permanent Oscillatory Cosmology is confirmed by 89 formula giving the Hubble radius, with 7 correlating to $10^{-9}$. The computer obtained the best formula using the Atiyah constant and the number 137, the Eddington's electric constant. This conforms with Atiyah's testimony about the Physics-Mathematics unification and the central role of arithmetics in this unification process. The identification with the Eddington statistical formula gives $G$, compatible with the $10^{-5}$ precise BIPM measurement and the $10^{-6}$ precise sun-quasar non-Doppler Kotov period. The hypothesis of a computing Cosmos implies a $\pi$ rationalization process which validates the Wyler's theory and the Fermion Koide formula in the $10^{-9}$ domain. Also the Eddington's equation is fully rehabilitated, by connecting the four forces.}
  % conclusions heading (optional), leave it empty if necessary 
   {}



   \maketitle
%
%-------------------------------------------------------------------

\section{Introduction}

   From Hirzebruch's work \cite{Hirzebruch}, which revolutionized \emph{geometry and topology\/}, Sir Michael Atiyah, Raoul Bott \cite{Bott} and Isadore Singer \cite{Singer} introduced the index theory, acclaimed by theoretical physicists \cite{Alvarez}. Following this path, on the advice of the physicist Gerard t'Hooft, Atiyah looked for the determination of the electrical constant $a \approx 137.035999085(21)$ \cite{Atiyah}.
    
    
    At the 2018 Heidelbergh Laureate Forum, he showed that the extrapolation of the Euler formula  $e^{2i\pi} = 1$ to the quaternions leads to the 'Atiyah constant' $\Gamma = \gamma a/\pi $. Meanwhile, he rehabilitated the Eddington \cite{Eddington} bare electrical constant, the prime number 137, and announced that the resolution of the Riemann conjecture appears as a "bonus". Moreover, the four forces would be connected to the four principal algebra, whose the octonion non-associative one would be tied to the gravitation constant $G$ in a future work \cite{Atiyah}.

%--------------------------------------------------------------------
\section{The scope and method}

%-------------------------------------- Two column figure (place early!)

   Quite independently, the $G$ value was tied to the invariant Hubble radius $R$ in the \textit {Coherent Cosmology}. A computer analysis has shown that it is confirmed in the ppb domain ($10^{-9}$) by simple formula involving the Atiyah constant \cite{Sanchez}. The aim of this article is to confirm the above \textit {arithmetic unificaton}.
   
   Our main hypothesis is that \textit{both} mathematics and physics standard model are widely incomplete, and that the so far unexplained measured adimensional constants (see Table 1) can be used as a guide for the overall arithmetics unification. Recall that the search for correlations between measurements is the heart of the scientific method, as the history shows, through Dalton, Proust, Balmer, Mendeleiv, Mandel... In particular, it was already shown that the Atiyah constant enters the core of Coherent Cosmology, the Topological Axis (Fig.1), both in connection with the Higgs boson and the galaxy group radius, a crucial cosmic distance \cite{Sanchez}. The pertinent data is completed by physical and cosmic constants in Tables 2 and 3.
   
   
   In conformity with this physics-mathematics unification idea, the Table 1 mixes physical adimensional constants \cite{Tanabashi} with pure mathematical constants. But among the later an important distinction is made. Only the whole numbers are considered really exact. For instance, the Archimedes constant $\pi$ is refered only as 'exact', meaning one can uses it in a computer calculation, only if one defines an imprecision domain. From this kind of argument, the Cosmos vastness has been justified by quasi-continuous quantum holography, where the whole large numbers of Lucas-Mersenne \cite{Bastin} and Eddington \cite{Eddington} are central \cite{Sanchez}.
      
      
   In particular, the fact that the Muon-Electron mass ratio is measured to 10 ppb, while nobody knows the role of Muon in Nature, is very intriguing. We show here that this permits to definitely validate the empiric, but dramatically simple, Koide fermion formula, connected with a rehabilited Wyler's theory through $\pi$ rationalisation process. 
   
   
   While the Atiyah's work does not seem to give the $a$ value, nor the Riemann conjecture solution, he suggested \textit{there is a bridge between the octonion algebra and the sporadic groups} \cite{Atiyah1}. This article will confirm this Atiyah's prediction, connecting these two appently separated mathematical domains.
   
   
   
   Note that the Topological Axis shows clearly the eight-fold Bott periodicity, typical of octonion algebra, and that Coherent Cosmology seems to involve the sporadic groups. This connection will be strengthened by a more attentive study of the modular function, clearly tied to the Monster group, the largest of the 26 sporadic ones \cite{Conway} \cite{Borcherds}.
   
   
   This article will show also the unexpected laison between the Topologcal Axis and the Periodic table of elements, and the height-manifold crystallography.
   


   \begin{figure*}
   \centering
   %%%\includegraphics{empty.eps}
   %%%\includegraphics{empty.eps}
   %%%\includegraphics{empty.eps}
   \caption{Adiabatic exponent $\Gamma_1$.
               $\Gamma_1$ is plotted as a function of
               $\lg$ internal energy $\mathrm{[erg\,g^{-1}]}$ and $\lg$
               density $\mathrm{[g\,cm^{-3}]}$.}
              \label{FigGam}%
    \end{figure*}
%
   In this section the one-zone model of \citet{baker},
   originally used to study the Cephe{\"{\i}}d pulsation mechanism, will
   be briefly reviewed. The resulting stability criteria will be
   rewritten in terms of local state variables, local timescales and
   constitutive relations.

   \citet{baker} investigates the stability of thin layers in
   self-gravitating,
   spherical gas clouds with the following properties:
   \begin{itemize}
      \item hydrostatic equilibrium,
      \item thermal equilibrium,
      \item energy transport by grey radiation diffusion.
   \end{itemize}
   For the one-zone-model Baker obtains necessary conditions
   for dynamical, secular and vibrational (or pulsational)
   stability (Eqs.\ (34a,\,b,\,c) in Baker \citeyear{baker}). Using Baker's
   notation:
   \[
      \begin{array}{lp{0.8\linewidth}}
         M_{r}  & mass internal to the radius $r$     \\
         m               & mass of the zone                    \\
         r_0             & unperturbed zone radius             \\
         \rho_0          & unperturbed density in the zone     \\
         T_0             & unperturbed temperature in the zone \\
         L_{r0}          & unperturbed luminosity              \\
         E_{\mathrm{th}} & thermal energy of the zone
      \end{array}
   \]
\noindent
   and with the definitions of the \emph{local cooling time\/}
   (see Fig.~\ref{FigGam})
   \begin{equation}
      \tau_{\mathrm{co}} = \frac{E_{\mathrm{th}}}{L_{r0}} \,,
   \end{equation}
   and the \emph{local free-fall time}
   \begin{equation}
      \tau_{\mathrm{ff}} =
         \sqrt{ \frac{3 \pi}{32 G} \frac{4\pi r_0^3}{3 M_{\mathrm{r}}}
}\,,
   \end{equation}
   Baker's $K$ and $\sigma_0$ have the following form:
   \begin{eqnarray}
      \sigma_0 & = & \frac{\pi}{\sqrt{8}}
                     \frac{1}{ \tau_{\mathrm{ff}}} \\
      K        & = & \frac{\sqrt{32}}{\pi} \frac{1}{\delta}
                        \frac{ \tau_{\mathrm{ff}} }
                             { \tau_{\mathrm{co}} }\,;
   \end{eqnarray}
   where $ E_{\mathrm{th}} \approx m (P_0/{\rho_0})$ has been used and
   \begin{equation}
   \begin{array}{l}
      \delta = - \left(
                    \frac{ \partial \ln \rho }{ \partial \ln T }
                 \right)_P \\
      e=mc^2
   \end{array}
   \end{equation}
   is a thermodynamical quantity which is of order $1$ and equal to $1$
   for nonreacting mixtures of classical perfect gases. The physical
   meaning of $ \sigma_0 $ and $K$ is clearly visible in the equations
   above. $\sigma_0$ represents a frequency of the order one per
   free-fall time. $K$ is proportional to the ratio of the free-fall
   time and the cooling time. Substituting into Baker's criteria, using
   thermodynamic identities and definitions of thermodynamic quantities,
   \begin{displaymath}
      \Gamma_1      = \left( \frac{ \partial \ln P}{ \partial\ln \rho}
                           \right)_{S}    \, , \;
      \chi^{}_\rho  = \left( \frac{ \partial \ln P}{ \partial\ln \rho}
                           \right)_{T}    \, , \;
      \kappa^{}_{P} = \left( \frac{ \partial \ln \kappa}{ \partial\ln P}
                           \right)_{T}
   \end{displaymath}
   \begin{displaymath}
      \nabla_{\mathrm{ad}} = \left( \frac{ \partial \ln T}
                             { \partial\ln P} \right)_{S} \, , \;
      \chi^{}_T       = \left( \frac{ \partial \ln P}
                             { \partial\ln T} \right)_{\rho} \, , \;
      \kappa^{}_{T}   = \left( \frac{ \partial \ln \kappa}
                             { \partial\ln T} \right)_{T}
   \end{displaymath}
   one obtains, after some pages of algebra, the conditions for
   \emph{stability\/} given
   below:
   \begin{eqnarray}
      \frac{\pi^2}{8} \frac{1}{\tau_{\mathrm{ff}}^2}
                ( 3 \Gamma_1 - 4 )
         & > & 0 \label{ZSDynSta} \\
      \frac{\pi^2}{\tau_{\mathrm{co}}
                   \tau_{\mathrm{ff}}^2}
                   \Gamma_1 \nabla_{\mathrm{ad}}
                   \left[ \frac{ 1- 3/4 \chi^{}_\rho }{ \chi^{}_T }
                          ( \kappa^{}_T - 4 )
                        + \kappa^{}_P + 1
                   \right]
        & > & 0 \label{ZSSecSta} \\
     \frac{\pi^2}{4} \frac{3}{\tau_{ \mathrm{co} }
                              \tau_{ \mathrm{ff} }^2
                             }
         \Gamma_1^2 \, \nabla_{\mathrm{ad}} \left[
                                   4 \nabla_{\mathrm{ad}}
                                   - ( \nabla_{\mathrm{ad}} \kappa^{}_T
                                     + \kappa^{}_P
                                     )
                                   - \frac{4}{3 \Gamma_1}
                                \right]
        & > & 0   \label{ZSVibSta}
   \end{eqnarray}
%
   For a physical discussion of the stability criteria see \citet{baker} or \citet{cox}.

   We observe that these criteria for dynamical, secular and
   vibrational stability, respectively, can be factorized into
   \begin{enumerate}
      \item a factor containing local timescales only,
      \item a factor containing only constitutive relations and
         their derivatives.
   \end{enumerate}
   The first factors, depending on only timescales, are positive
   by definition. The signs of the left hand sides of the
   inequalities~(\ref{ZSDynSta}), (\ref{ZSSecSta}) and (\ref{ZSVibSta})
   therefore depend exclusively on the second factors containing
   the constitutive relations. Since they depend only
   on state variables, the stability criteria themselves are \emph{
   functions of the thermodynamic state in the local zone}. The
   one-zone stability can therefore be determined
   from a simple equation of state, given for example, as a function
   of density and
   temperature. Once the microphysics, i.e.\ the thermodynamics
   and opacities (see Table~\ref{KapSou}), are specified (in practice
   by specifying a chemical composition) the one-zone stability can
   be inferred if the thermodynamic state is specified.
   The zone -- or in
   other words the layer -- will be stable or unstable in
   whatever object it is imbedded as long as it satisfies the
   one-zone-model assumptions. Only the specific growth rates
   (depending upon the time scales) will be different for layers
   in different objects.

%--------------------------------------------------- One column table
   \begin{table}
      \caption[]{Opacity sources.}
         \label{KapSou}
     $$ 
         \begin{array}{p{0.5\linewidth}l}
            \hline
            \noalign{\smallskip}
            Source      &  T / {[\mathrm{K}]} \\
            \noalign{\smallskip}
            \hline
            \noalign{\smallskip}
            Yorke 1979, Yorke 1980a & \leq 1700^{\mathrm{a}}     \\
%           Yorke 1979, Yorke 1980a & \leq 1700             \\
            Kr\"ugel 1971           & 1700 \leq T \leq 5000 \\
            Cox \& Stewart 1969     & 5000 \leq             \\
            \noalign{\smallskip}
            \hline
         \end{array}
     $$ 
   \end{table}
%
   We will now write down the sign (and therefore stability)
   determining parts of the left-hand sides of the inequalities
   (\ref{ZSDynSta}), (\ref{ZSSecSta}) and (\ref{ZSVibSta}) and thereby
   obtain \emph{stability equations of state}.

   The sign determining part of inequality~(\ref{ZSDynSta}) is
   $3\Gamma_1 - 4$ and it reduces to the
   criterion for dynamical stability
   \begin{equation}
     \Gamma_1 > \frac{4}{3}\,\cdot
   \end{equation}
   Stability of the thermodynamical equilibrium demands
   \begin{equation}
      \chi^{}_\rho > 0, \;\;  c_v > 0\, ,
   \end{equation}
   and
   \begin{equation}
      \chi^{}_T > 0
   \end{equation}
   holds for a wide range of physical situations.
   With
   \begin{eqnarray}
      \Gamma_3 - 1 = \frac{P}{\rho T} \frac{\chi^{}_T}{c_v}&>&0\\
      \Gamma_1     = \chi_\rho^{} + \chi_T^{} (\Gamma_3 -1)&>&0\\
      \nabla_{\mathrm{ad}}  = \frac{\Gamma_3 - 1}{\Gamma_1}         &>&0
   \end{eqnarray}
   we find the sign determining terms in inequalities~(\ref{ZSSecSta})
   and (\ref{ZSVibSta}) respectively and obtain the following form
   of the criteria for dynamical, secular and vibrational
   \emph{stability}, respectively:
   \begin{eqnarray}
      3 \Gamma_1 - 4 =: S_{\mathrm{dyn}}      > & 0 & \label{DynSta}  \\
%
      \frac{ 1- 3/4 \chi^{}_\rho }{ \chi^{}_T } ( \kappa^{}_T - 4 )
         + \kappa^{}_P + 1 =: S_{\mathrm{sec}} > & 0 & \label{SecSta} \\
%
      4 \nabla_{\mathrm{ad}} - (\nabla_{\mathrm{ad}} \kappa^{}_T
                             + \kappa^{}_P)
                             - \frac{4}{3 \Gamma_1} =: S_{\mathrm{vib}}
                                      > & 0\,.& \label{VibSta}
   \end{eqnarray}
   The constitutive relations are to be evaluated for the
   unperturbed thermodynamic state (say $(\rho_0, T_0)$) of the zone.
   We see that the one-zone stability of the layer depends only on
   the constitutive relations $\Gamma_1$,
   $\nabla_{\mathrm{ad}}$, $\chi_T^{},\,\chi_\rho^{}$,
   $\kappa_P^{},\,\kappa_T^{}$.
   These depend only on the unperturbed
   thermodynamical state of the layer. Therefore the above relations
   define the one-zone-stability equations of state
   $S_{\mathrm{dyn}},\,S_{\mathrm{sec}}$
   and $S_{\mathrm{vib}}$. See Fig.~\ref{FigVibStab} for a picture of
   $S_{\mathrm{vib}}$. Regions of secular instability are
   listed in Table~1.

%
%                                                One column figure
%----------------------------------------------------------------- 
   \begin{figure}
   \centering
   %%%\includegraphics[width=3cm]{empty.eps}
      \caption{Vibrational stability equation of state
               $S_{\mathrm{vib}}(\lg e, \lg \rho)$.
               $>0$ means vibrational stability.
              }
         \label{FigVibStab}
   \end{figure}
%-----------------------------------------------------------------


\section{The cosmic liaison between $a$ and the weak mixing angle}

Thus, the physical parameters would be mathematical constants of an unknown arithmetical domain. So, their « fine-tuning» is not due to hazard in a Disparate Multiverse, but are of mathematical origin in a single Cosmos unifying coherent universes. One main result of Coherent Cosmology is that the Cosmos volume, with length unit the Hydrogen radius $r_H$, involves $a^a$, showing that \textit{$a$ is an optimal computation base}  \cite{Sanchez}:

\begin{equation}
    (4\pi /3) (R_C/r_H)^3 \approx a^a/\pi \approx (1/ln2)^{\sqrt{pH}} \approx (13/3)^{p/4} \approx (1/sin^2\theta)^{n/4} 
\end{equation}

where $p$, $H$ and $n$ are the proton-electron, hydrogen-electron and neutron-electron mass ratios, and 13/3 the fraction associated to the decomposition 16 = 13 + 3 \cite{Sanchez1}. This corresponds to the value $sin^2\theta \approx 0.231235$, compatible with the measured weak mixing angle 0.21322(4) \cite{Tanabashi}. Note that the presence of $ln2$ invoves information theory \cite{Shannon}. 

The \textit {effective} mixing angle 0.21355(4) \cite{Tanabashi} is compatible with $ln^2\Phi \approx 0.2315648$. The formulas n. 15 and 16 of Table 5 tying $R$ to $\Phi$ and $ln\Phi$ implies the following ppm correlation: 

\begin{equation}
    \Phi^{\sqrt e} \approx (1/ln\Phi)^{\sqrt {2\pi}} (a/137)
\end{equation}

Analysis shows this ppb expression for a-137

\begin{equation}
   a - 137 \approx H ln\Phi /eplna
\end{equation}

so the pertinence of $ln\Phi$ cannot be doubted. It is significative that no mathematics domain includes it.
  
\section{The Rehabilitation of Wyler\'s theory}
The presence of an excess $\pi$ in the above formula suggests that $\pi$ is also a computation base for the Cosmos.This is indeed the case in the even Riemann series.


Atiyah did not consider this computation point of view, but insisted on the analogy of his procedure with that of Archimedes for calculing $\pi$ \cite{Atiyah}. But, \textit {in the hypothesis that the cosmos is a computer}, the cosmos cannot use the mathematical Archimedes constant $\pi$, which is an idealisation, \textit {otherwise any time calculation would be infinite}. Its decomposition is an unresolved problem, but the first terms are : $3, 7, 16, -293.634$, where the fourth term, hightly singular, is so close (3 ppm) to $1 + n/2\pi$, where $n$ is the neutron/electron mass ratio.

\begin{equation}
\pi : 3, 7, 16, -(1+n/2\pi)
\end{equation}


In the famous Wyler formula \cite{Wyler} 

\begin{equation}
(3\sqrt a/4)^8 = 120 \times \pi_W^{11}
\end{equation}


implicitely tied to the 11D supergravity space, the development of  $\pi_W$ shows an analogy with the above one, apart the insertion of \textit {the singular prime 163} :

\begin{equation}
\pi_W : 3, 7, 16,- 163/2, -(1+n/4\pi) ~~~~\Rightarrow ~~~~    a \approx 137.03599908399
\end{equation}


This number 163 is the last of the Heegner-Stark numbers \cite{Stark}. 


Moreover, according to Atiyah \cite{Atiyah1}, an approximation of $\pi$ appears directly in $(a^2-137^2)^{1/2} = \pi_{a,137} : 3, 7, 10, a_s$, , where the forth term is very close to the inverse strong coupling constant $1/a_s \approx 0.1179(10)$ \cite{Tanabashi}. The proton-electron mass ratio of Wyler \cite{Wyler} is the simple formula 

\begin{equation}
 p_W = 6\pi^5    
\end{equation}

which is \textit {the product of the area of a cube of side $\pi$ with its volume}. Taking the above value $\pi_W$, this gives $p_{W, a, 137} = 6\pi_a^5 \approx 1833.99893 \approx\ p_W/d_e$, which is of central pertinence in cosmology: indeed it connects both with $R$ and $R_1$, the one-electron universe radius \cite{Sanchez} (Table 4).

Note that the fractionnal development of e swhows also a direct connexion with physics: $e : 2, 7/5, 3\sqrt{p_G} where p_G = P/\sqrt {N_L}$. Also, the formula 24 and 25 of table 5 implies the 10 ppm connexion:

\begin{equation}
 e \approx 1 + e^{2/e^2}    
\end{equation}

From the proximity of $a + 2\pi$ with the Planck law term $e^w$, it was investigated if $a$ was a trigonometric line. Indeed 
 
\begin{equation}
cosa = 1/e_a ~~~~~~~~  e_a : 2, 7/5, ee_a \sqrt{dn/2\pi} 
\end{equation}

So, it is proven that Nature uses approximations of the main mathematical constants $\pi$ and $e$.

\section{The Rehabilitation of Eddington\'s Equation}
Eddington proposed the following equation, for which the ratio of the roots would give the proton-electron mass ratio:

\begin{equation}
 10x^2 - 136x +1 = 0     
\end{equation}

this ratio is $p_E \approx 1847.599459$, very close to $(p/H)(4\pi_E)^2\sqrt a$. The corresponding $\pi_E$ showing a special development : 3, 7, y, where y is the root of the equation $6y^2 - 99 y + 16 = 0$. 

The roots $x_1$ et $x_2$ of this equation are noted in Table 1: $x_1 approx 13.5926430$ and $1/x_2 = 10 x_1 \approx  135.926430$. Now, with the optimised value of the inverse strong coupling constant $a_s$ (table2):

\begin{equation}
 1/x_2 = 10 x_1 \approx 16 a_s e^{1/139}     
\end{equation}

meaning the 139th root of $e$ is involved. This means that $a^a$ is central, see below, and that is $x_2$ directly tied to the strong coupling. The first root is directly tied to the weak coupling, by the 3 ppm formula:

\begin{equation}
  x_1 \approx 2F^2 \beta /137\times a^4    
\end{equation}

Moreover, as $x_1 \approx 2^{2^9/136}$, it connects with the critical radius $R$ (formula n. 23 in table 5), implying a liaison with gravitation. \textit{So the 4 forecs are resumed in the Eddington's equation}.

\section{The Central Role of the Modular Number $j_0$ = 744}

The \textit{Ramanujan quasi-whole number} $N_R = exp(\pi\sqrt(163))$, is tied to the Dedekind eta function, which plays a role in bosonic string theory \cite{Apostol}\cite{Lovelace}, wholly rehabilitated by the Topological Axis (Fig. \href{fig:7:fig1}).


This large number is also tied to the modular function $j_m$, whose Fourier series shows linear combinations of dimensions of the irreductive representations of the Monster group. With $q = 2\pi x$ :

\begin{equation}
j_m(x) = 1/q + 744 + 196884 q + ...
\end{equation}

In particular 196884 = D + 1 where D = 196883 is the Monster group order. This was called the Monster Moonshine \cite{Conway}. It was shown that string physics makes a bridge between these two separated mathematical domains \cite{Borcherds}, but the connexion with octoinons is not observed. 


But the q-independent number number $j_0$ = 744 is unexplained. It is directly connected to the Monster group order:

 \begin{equation}
 \pi/2 \approx lnlnlnO_M \approx 1/lnlnlnj_0
\end{equation}

Moreover the forth natural logatithm of $O_M$ shows a double correlation:

 \begin{equation}
 lnlnlnlnO_M \approx lnO_M /(2\times 137) \approx e/6
\end{equation}

while the Baby Monster group order appears in:

\begin{equation}
 O_B \approx j_0^{\sqrt{137}}
\end{equation}
see below the pertinence of deviation from this formula

$j_0$ enters the simplest couples of all formulas (n. 15 and 16 in 
 Table \ref{tab:5:table5} )
 
 \begin{equation}
 R/\lambdabar_e \approx j_0^{n/a} \approx j_0^{2eap\beta/j_0} 
\end{equation}

So the most elementry cosmic test shows a symmetry proton-electron in the ppm relation:

\begin{equation}
j_0 \approx 2eap\beta/n 
\end{equation}


It is related to the above fraction 13/3 by the relation involving the Monster and the bosonic string dimension 26:

\begin{equation}
  O_M^{1/26^2} \approx j_0^{1/6^2}\approx (1/ln(1+1/d_e))^{1/2}   
\end{equation}

where $d_e \approx 1.001159$ is the electron magnetic moment excess \cite{Tanabashi}.
 Moreover, there is a tight liaison with the Baby Monster group order and the Eddington's brut value 136:
 
 \begin{equation}
 ( O_B/ j_0^{\sqrt{137}})^2 \approx a -136 \approx 1 + 1/\sqrt{j_0}
\end{equation}

$j_0$ connects also with the topological function $f(d ) = exp(2^(d/4))$, the Sternheimer scae factor $j$, and the single electron universal radius $R_1$ \cite{Sanchez}
 
 \begin{equation}
 exp(2^(\sqrt{j_0}/4)) \approx e^{j -1} \approx O_M/(2\times 136)^2 \approx R_1 \lambdabar_e / \lambdabar_W \lambdabar_Z
\end{equation}

Writing $\sqrt{j_0}$ as a dimension $2+4k$, this shows that $k \approx 2 \pi_{Egy}$, where $\pi_{Egy} = (4/3)^4$. This leads to the 14 ppb relation, where $\tau$ is the Tau-Electron mass ratio: 

\begin{equation}
 \sqrt{j_0} \approx 2 (1+4\pi_{Egy}) - 1/137 -1/\tau
\end{equation}
 

Moreover $j_0 = (3/2) \times 496$, where 496 is the dimension of the superstring gauge group SO(32), a necessary conditions for a superstring theory to make sense \cite{Green}.


But 10-dimensional string theory is the version of the theory that uses octonions algebra \cite{Schlay}. So, it seems that the Atiyah's conjecture was correctly prophetic.

\section{The Decisive Role of the term $a^a$ }

The product of the 6 pariah sporadic groups is directly tied with $a^a$ and $F/a$, $F$ being the ratio Fermi/electron \cite{Sanchez}. 

Moreover, $a^a$  is also very close to the Lucas-Lehmer term $S_9 = g_3^{2^9}$, where $ g_3 = 2 + \sqrt(3)$ is the generator of quasi-whole numbers. Now, as recalled before, the Lucas-Mersenne Large Number $N_L = 2^{127} - 1$ plays a central role in Coherent Cosmology \cite{Sanchez}. It is prime because it is a divisor of the huge number $S_{125}$, which appears to connect also in cosmology (last formula of Table \ref{tab:4:table4}).


Now, $a^a$ connects also directly with the famous Ramanujan quasi-whole number $N_R = exp(\pi \sqrt(163))$, tied to the above Heegner-Stark number 163, which exhibits staggering correlations:



\begin{equation}
lnR_N = \pi \sqrt(163)  \approx lna \times ln\tau  \approx  lnp \times ln\mu
\end{equation}

\begin{equation}
a^a \approx N_R^{\tau/\mu} 
\end{equation}

\begin{equation}
\tau/\mu   \approx g(1) \approx  2a_s \approx \sqrt {N_{ph}^(1/3)/n_{ph}}
\end{equation}

where $g(k) = exp(2^k)/k$ is the Topological Function (Figure \ref{fig:7:fig1}), while $p$, $\mu$ and $\tau$ are respectively the masses of Proton, Muon, and Tau by respect to the Electron one.  Now they seem to be related to Topological Axis Function $g(1)$ and the strong coupling $a_s$. The dramatic intervention ($10^{-3}$ precision) of the cosmic number of photons $N_{ph}$ and the universal one $n_{ph}$ confirms de whole procedure


Moreover, the connection is made with the Riemann series $\zeta(3)$:

\begin{equation}
alna/ln\zeta(3) \approx p_G \approx p_{W,a,137}/d_e \approx p_W/d_e^2
\end{equation}

confirming the above relation:$d_e \approx p_W /p_{W, a, 137}$

\section {The Koide-Wyler ppb relation}

While $\mu$ is measured to 10 ppb, $\tau$ is rather badly measured. The Koide relation \cite{Koide}, always unexplained, has shown correct predictability for the $\tau$ mass, proving \textit{the present standard particle theory is badly insufficient}. This relation writes in the most symmetrical way connecting with the above Wyler formula, in the following \textit{ppb formula, which confirms the specified ppb value}: $\mu = (Fa/\sqrt{pH})^{1/2}$\cite{Sanchez}.


\begin{equation}
(1 + \mu + \tau)/2 = (1 + \sqrt{\mu} + \sqrt{\tau})^2/3 = p_K \approx 6\pi_K^5 (1+(\mu / \tau)^2) ~~~~\pi_K: 3,7,16,-(2\times 137)^{2/3}
\end{equation}

he fermions Mu and Tau are complete mystery in the standard model. However, Eddington predicted the tau fermion, 35 years before its fascinating discovery, calling it « heavy Mesotron », with a right order prediction of its mass \cite{Eddington}.  This was very surprising because the Eddington theory, accused of pythagorism, has been the subject of great denial.

\section{The rehabilitation of 137 from the electroweak coupling constant}

But Eddington also predicted the importance of the $N_{16} = 136$ elements of a symmetrical matrix 16 x 16, giving 137 by adding unity, whose pertinence is confirmed by the very precisely (0.1 ppm) measured electroweak coupling (inverse) factor 

\begin{equation}
a_w = (2\times137 \Gamma)^3
\end{equation}
 
Atiyah presented this number by the form $137 = 2^0 + 2^3 + 2^7$. Moreover, this additive unity is clearly tied to the Combinatorial Hierarchy \cite{Bastin}, based on the Catalan-Mersenne series starting with 3, because $N_4 = 10 = 3 + 7 and 3+7 + 127 = 137 = N_{16} + 1$. The following term $N_{32}  = 528$ cannot be compared with the huge Lucas-Mersenne Large Number $2^{127}- 1$, so it is the last term of the Hierarchy. 

This Lucas number appears in the ppb precise formula of Table \ref{tab:3:table3}, in liaison with 137. Moreover $32^2 - N_{32} =  496$ is the above dimension of the superstring gauge group SO(32), and \textit {the third perfect number}, see below the paramount importance of this fact, unrecognized up to this day.




\section{The Planck law connection with the Bernouilli function}

Now $a$ is tied in the superstring 9D space with the two constants of the Planck law, whose kernel is the Bernouilli fonction, $x/(1-e^{-x})$, \textit {central in the Atiyah's work} \cite{Atiyah}. These are the reduced Wien displacement constant $w$, and the number of photons $16\pi \zeta(3)$ in a volume $\lambda^3$, with $\lambda = hc/kT $, corresponding to one photon by volume $l_{ph}^3$:

\begin{equation}
(16\pi\zeta(3))^3/w^4 = \lambda^5l_{Wien}^4/l_{ph}^9 \approx \pi_a^3a    \Rightarrow     \pi_a: 3;7;16;17p_an/p
\end{equation}

As in the preceeding case, this is a symbolic rationalisation. This is the single formula obtained by computer in this article, exept the ppb precise relations involving the Atiyah constant in the Table \ref{tab:6:table6}.


\section{The Holographic Fine-Tuning with the Hubble Universal Critical Radius}

Among the 30 or so free parameter of the present standard model, the Nature seems to favor some ones (Hierarchy Principle \cite{Sanchez}). They distinguish themselves as being measured with high precision, so the Table \ref{tab:2:table2} does not include the quarks, neither the neutrinos. 

Recall that the Cosmos seems to be ruled by the Holographic Principle and its Diophantine form, the Holic Principle, presented in 1994 at ANPA (Cambridge)  \cite{Sanchez1}. Orsay University gave a sabbatical year (1997-1998) to F.M. Sanchez, in order to develop the application of the Holographic and Holic principle in theoretical physics. In the three first minutes of this sabbatical year, Francis M. Sanchez found, by the most elementary method, based on the universal constants, half the length 13.80(2) Gly (billion light-years). This was deposed in a closed letter in March 1998 at the French Academy. 


So, to show that the Hubble radius is constant, it was sufficient, in elementary dimensional analysis, to replace the speed $c$ by the mean masses of the 3 main particles in Atomic Physics. Note that the general use of $c$ = 1 seems to have precluded this discovery before. Also, for most theorists, the proton is not a so fundamental particle as the electron. But this is a reductionist point of view. In fact, the proton mass is fairly well measured (Table \ref{tab:2:table2}), while the quark masses are not, as recalled above.
 
 
    This 2 factor is typical of the critical Schwarzschild radius $2 = Rc^2/GM$, and is also presented in the Archimedes testimony, as the ratio between the perimeter and the area of a disk with radius unity, \textit {as expressively noticed by Atiyah }. It was the first historic holographic relation. So, the critical radus is given by an holographc relation defining a space quantum $l_0$:
    
    \begin{equation}
        \pi (R/l_P)^2 = 2\pi R/l_0
    \end{equation}
    
    So this universal radius $R$ may be considered as the radius for which, in an homogeneous Universe (the basic cosmological principle), the included mass reaches the above critical value\cite{Sanchez}. Thus each space quantum (topon) in the cosmos is the center of a sphere with universal radius $R$.  
    
    
    This permits to resolve the question of the enormity of the vacuum energy by pushing down the Plank wall by a factor $10^61$, resolving also the vacuum quantum energy dilemna  \cite{Sanchez} . 
    
    
    At the same epoch, some theorists, as t' Hooft \cite{Hooft}, introduced also the Holographic Principle, but could not apply it to the Universe, believing the Hubble radius is variable.
    
    
    In fact these authors applied the above disk area to a blackhole, calling it 'Bekenstein-Hawking entropy'\cite{Bekenstein}, but, instead of considering the disk, they considered the sphere area, introducing the useless factor 1/4. 
    
    
    In fact it was shown that, starting from the real disk, a 3D sphere surface can be generated by rotating it around a diameter, leading, via an universal quantification tying the electron to the Lucas Number and the proton to the Eddington Number \cite{Sanchez}. \textit {This explains why the cosmos is so large}. Indeed, it tries to mimic a continuous space, to use approximations of $\pi$ in  the calculation.
    
    
    The critical factor 2 can be also considered as the ratio between the areas of a unit-radius sphere to the circonference of diametral disk. The extension to the 3D volume gives the nominal Cosmic Microwave Background (CMB) wavelength, corresponding to 2.73 Kelvin, in function of atomic and molecylar hydrogen wavelengths:\cite{Sanchez}:
    
    \begin{equation}
        2\pi R/\lambdabar_e \approx 4\pi (\lambdabar_H/l_P)^2 \approx (4\pi/3) (\lambdabar_{CMB}/\lambdabar_{H2})^3
    \end{equation}

    The series of 69 formula presented in this article confirm the invariance of both the Hubble radius and the cosmos temperature, as well as the background (cosmos) temperature. Let us recall that the Hubble radius is defined by $R = c/H_0$, where $H_0 = v/d$ is the Hubble constant, which implies the apparent speed $v$ in the red-shift of a $d$-distant galaxy$ v = c \Delta \lambda/\lambda$. So, there is \textit {the direct simpler relation} $\Delta \lambda/\lambda =  d/R$. 
    

    
    The so-called standard Universe age is 13.80(2) billion years \cite{Tanabashi}, while the Hubble radius deduced from the super novae 1a is $R_{SN1a} \approx 13,6(6)$ Gly \cite{Freeman}\textit{This article shows this cannot be related to any age}, since this length is given by a series of 69 formula implying invariant quantities, including the cosmic background in 9 formula . 
    
    
    This recalls 14 formulas presented by Jean Perrin in 1909 to prove definitely the real existence of atoms. Here, the task is to show the existence of an ultimate theory of massive strings in a dramatic re-interpretation of standard cosmology:\textit{ the Big Bang becoming permanent, and the Multiverse becoming coherent: each point is the center of a $R$-radius sphere}. This means that the Universe is destroyed and reconstructed in an high-frequency oscillation. This permits to consider matter as an matter-antimatter oscillation \cite{Sanchez2}. 
    
    This opens to the possibility that \textit {dark matter, whose existence is proven} by the connection with the Eddington large number $N_{Edd}$, (table 1), would be a quadrature oscillation.
    
    \section{The connection with Diophantine and Eddington physics}

These formulas give a radius $R$ value compatible with the following Diophantine analysis. The movement $(r,v)$ of a mobile in a gravitational central field has the form $r v^2 = Gm_G$, where $m_G$ is a characteristic mass. Viewing the third Kepler law as a Diophantine one,i.e. only resolvable in whole numbers, it resolves in  $T^2 = L^3 = n^6$, thus $L = n^2$,the orbital law in the Hydrogen atom, characterized by $rv = \hbar/m_{\hbar})$. So, \textit{there is a kind of symmetry between $G$ and $\hbar$}. Consider the following system, using the two principal masses, the electron and proton's ones: 

\begin{equation}
  r v^2 = Gm_e
  \end{equation}
  \begin{equation}
r v = \hbar/m_p  
\end{equation}

Thus, with the Planck mass $m_P = (\hbar c/G)^{1/2}$ : 
\begin{equation}
c/v = m_P^2/m_em_p = \sqrt(M/m_e)~~~~~~~~   M = m_P^4/m_em_p^2
\end{equation}

By identifying this mass with the critical mass of the Universe, this is the statistical solution \cite{Durham} of the Large Number Question by Eddington  : $R = 2 \sigma \sqrt{(M/m_0)}$, where the reference mass $m_0$ is identified to $m_e$ and the standard deviation $\sigma$ to $\hbar/csqrt{m_pm_H}$, in conformity with the gravitational Hydrogen molecule model \cite{Sanchez}. The optimized value for $G$ follows:
\begin{equation}
R = 2\hbar^2/Gm_em_pm_H  ~~~~  \Rightarrow G \approx 6.67545375 \times 10^{-11}  kg^{-1}m^{3}s^{-2}    
\end{equation}

which is compatible with the BIPM value \cite{Quinn}, precise to 10 ppm, but not with the standard value \cite{Tannabashi} which is the incongruous mean between discordant measurements. Moreover, this $G$ value is compatible with the value corresponding to the elimination of c between the gravitational and electroweak coupling constants (among the last formula of Table \ref{tab:2:table2}), leading to specify the non-Doppler quasar Kotov period $t_K \approx 9600.591457$ sec \cite{Sanchez}.


\section{The specified connection with the topological function}

Using the Holographic Principle, the cosmic quantities associated to this critical radius R are defined in the Table \ref{tab:5:table5}, in particular the Cosmos radius, which shows a dramatic connection with the topological term $g(7)$ :

\begin{equation}
R_{C}/\lambdabar_{e} g(7) \approx  \lambdabar_{e}/6l_P \approx F^5/6a^3 \approx (\lambdabar_{CMB}/r_H)^3 \approx  (am_p/m_e)^4  
\end{equation}

This induces the discovery of the Central Gravito-Electroweak relation :

\begin{equation*}
F^5/a^3 \approx \eta P    
\end{equation*}{}
  

with  $F = (2\times 137 \Gamma)^{3/2}$, the Fermi-Atiyah-Sanchez, factor, specifying the measured value of  $F$ (Table 1) with the help of the above Atiyah's constant $G = a \gamma/ \pi$, where appears the factor $\eta = 419/417$, very close to the Sternheimer limma $2^{1/144}$ \cite{Sternheimer}], which will be conneted to 10D cristallography below.


Moreover, this confirms that the Cosmos is the real source of the background radiation [5]:
\begin{equation}
F^5  \approx 6(\lambdabar_{CMB}/\lambdabar_{e})^3 \Rightarrow  T_{CMB}  \approx  2.725820 K  (mes : T_{CMB}  \approx  2.7255(6) K)) 
\end{equation}


The graviton mass, calculated from the double step holo-tachyonic propagation is associated with that of the photon. This graviton mass connects directly with $g(6)$ :

\begin{equation}
m_N/m_{gr} \approx g(6)/(1+1/\mu)^2 \Rightarrow    t_K  \approx  9600.65 sec ~~(mes : t_K \approx 9600.60(1) sec)    
\end{equation}

implying the mass ratio Muon-Electron. 


\section{The connection with the Periodic Table of Elements}

     In fact the pythagorism is in accordance with a quantum computation world ruled by Arithmetics. In particular the four smaller dimension numbers of the Topological Axis (Fig. 1) : 2, 6, 10, 14 identify with the atomic numbers of the Periodic Table spectroscopic series : $s, p, d, f$ . The Periodic Table contains 19 such series, corresponding to 118 atoms : 7s + 6p + 6d + 2f = 118 (atomic number of the Oganesson nucleus). 
     

     Now the periods are distinct from the principal quantum number, so that the periods starting from the second one are double. So, the above number of atoms decompose in $118/2 = 59 = 1 + 3s + 3p + 2d + f$.  By separating the last series f + 1 = 15, the theoretical decomposition 137 = 107 + 30 is justified by the sum $137 = 7(s +1) + 6(p +1) + 4(d +1) + 2(f +1)$. Note that $s + 1 = 3$ and $p + 1 = 7$ are the first terms of the above Combinatorial Hierarchy\cite{Bastin} .



     Consider all the series in the Topological Axis, by introducing the supplementary series $g, h, i, j$ of dimensions 18 ; 22 ; 26 ; 30, corresponding to the higher part of the Topological Axis, after the 16 which is the central dimension, this leads to
     
     \begin{equation}
      8s + 7p + 6 d + 5f + 4g + 3h + 2i + j = 408 = 3 \times 136   
     \end{equation}
      
     
     This writes, in function of the 10 D point symmetry operation numbers :  $k_{10-} = 165$ and $k_{10+} = 419$: $SO3 \times 136 = 419 - 11 = 165 + 35$, and $419-165 = 2 \times 127 =  35 + 11$. Note that the later is the supergravity dimension number and that $128/3^5$ is the classical musical limma. 
     
     
     But the superstring theory is only coherent in 9D space. For every odd dimension number, $k_{(2n - 1)-} = k_{(2n - 1)+} = k_{2k-}$ so the above combination type $k_- + 2k_+$ is for 9D: $3 \times 165 = 495$, the canonical reduced number attached to the above perfect number 496. This number 495 is associated to the Higgs boson (Fig. 1) and to the smallest sporadic group, the Mathieu one, of order 16×495. Note that the couple 495-496 has the same Euler index (240) and the same Carmichael-lambda index (60). This could be unique, defining 496 as a super-perfect number. Note also that 496 is close to the 20th root of the Monster order.
     
 \section{Connections with the high-dimensional crystallography}
 
 Now the above numbers 19 and 59 are the Crystalline Ponctual Symmetry Operation numbers ($PSO_{Cr}$), respectively negative and positive in 6D crystallography \cite{Weigel} (Table 7) : $k_{6-} = 19$, $k_{6+} = 59$. Note that this dimension d = 6 corresponds to k = 1 in the Topological Axis. So the above definition of 137 writes:
  \begin{equation}
   137 =   = 7(s +1) + 6(p +1) + 4(d +1) + 2(f +1) = k_{6-} + 2k_{6+}   
    \end{equation}
    which is also:
    \begin{equation}
    K_{5+}+K_{6+}+K_{7+} = K_{6-} + K_{7-} +K_{8-} = 137   
    \end{equation}
    while:
    \begin{equation}
    K_{10+} +K_{11+}+K_{12+} = K_{11-} +K_{12-}+K_{13-} = 1839   
    \end{equation}
    
    this number 1839 is the whole number closest to the neutron-electron mass ratio.
    
    Moreover, the sum of the mean values $K_{d} = (K_{d+} + K_{d-})/2$ untill the dimension $d = 12$ is 1836 (Table \ref{tab:7:table7}), which is the entire part of the proton-electron mass ratio. Note that this sum limited to $d = 7$ gives 138. So it seems that the dimension $12$ will play a role in the future string theory.
    
    
    Note that in the  above ratio 419/417, the number $419$ is the number of positive Point Operation in 10D cristallography, while $417$ is the number of trivial ones .
    
    
    Note that the roots of the crystallographic algebraic equation of degree $n$ are of type $exp(i2\pi m/l)$, where $l$ and $m$ are whole numbers such that $ l \leq n $ and $ 1 \leq m \leq l $ ,  : this is similar to the above spectroscopic series. Such an unexpected connection needs also further study.   



\section{Conclusions}

   \begin{enumerate}
      \item The conditions for the stability of static, radiative
         layers in gas spheres, as described by Baker's (\citeyear{baker})
         standard one-zone model, can be expressed as stability
         equations of state. These stability equations of state depend
         only on the local thermodynamic state of the layer.
      \item If the constitutive relations -- equations of state and
         Rosseland mean opacities -- are specified, the stability
         equations of state can be evaluated without specifying
         properties of the layer.
      \item For solar composition gas the $\kappa$-mechanism is
         working in the regions of the ice and dust features
         in the opacities, the $\mathrm{H}_2$ dissociation and the
         combined H, first He ionization zone, as
         indicated by vibrational instability. These regions
         of instability are much larger in extent and degree of
         instability than the second He ionization zone
         that drives the Cephe{\"\i}d pulsations.
   \end{enumerate}

\begin{acknowledgements}
      Part of this work was supported by the German
      \emph{Deut\-sche For\-schungs\-ge\-mein\-schaft, DFG\/} project
      number Ts~17/2--1.
\end{acknowledgements}

% WARNING
%-------------------------------------------------------------------
% Please note that we have included the references to the file aa.dem in
% order to compile it, but we ask you to:
%
% - use BibTeX with the regular commands:
%   \bibliographystyle{aa} % style aa.bst
%   \bibliography{Yourfile} % your references Yourfile.bib
%
% - join the .bib files when you upload your source files
%-------------------------------------------------------------------

\begin{thebibliography}{}

  \bibitem[Baker(1966)]{baker} Baker, N. 1966,
      in Stellar Evolution,
      ed.\ R. F. Stein,\& A. G. W. Cameron
      (Plenum, New York) 333

   \bibitem[Balluch(1988)]{balluch} Balluch, M. 1988,
      A\&A, 200, 58

   \bibitem[Cox(1980)]{cox} Cox, J. P. 1980,
      Theory of Stellar Pulsation
      (Princeton University Press, Princeton) 165

   \bibitem[Cox(1969)]{cox69} Cox, A. N.,\& Stewart, J. N. 1969,
      Academia Nauk, Scientific Information 15, 1

   \bibitem[Mizuno(1980)]{mizuno} Mizuno H. 1980,
      Prog. Theor. Phys., 64, 544
   
   \bibitem[Tscharnuter(1987)]{tscharnuter} Tscharnuter W. M. 1987,
      A\&A, 188, 55
  
   \bibitem[Terlevich(1992)]{terlevich} Terlevich, R. 1992, in ASP Conf. Ser. 31, 
      Relationships between Active Galactic Nuclei and Starburst Galaxies, 
      ed. A. V. Filippenko, 13

   \bibitem[Yorke(1980a)]{yorke80a} Yorke, H. W. 1980a,
      A\&A, 86, 286

   \bibitem[Zheng(1997)]{zheng} Zheng, W., Davidsen, A. F., Tytler, D. \& Kriss, G. A.
      1997, preprint
\end{thebibliography}

\end{document}
%
%%%%%%%%%%%%%%%%%%%%%%%%%%%%%%%%%%%%%%%%%%%%%%%%%%%%%%%%%%%%%%
Example below of non-structurated natbib references  
To use the v8.3 macros with this form of composition of bibliography, 
the option "bibyear" should be added to the command line 
"\documentclass[bibyear]{aa}".
%%%%%%%%%%%%%%%%%%%%%%%%%%%%%%%%%%%%%%%%%%%%%%%%%%%%%%%%%%%%%%

\begin{thebibliography}{}

\bibliographystyle{unsrt}  
%\bibliography{references}  %%% Remove comment to use the external .bib file (using bibtex).
%%% and comment out the ``thebibliography'' section.
%%% Comment out this section when you \bibliography{references} is enabled.
% \begin{thebibliography}{99}
\bibitem{Hirzebruch} Hirzebruch F. Topological methods in algebraic geometry. Springer 1966.\\
\bibitem{Bott} M. Atiyah, R. Bott, V. Patodi, "On the heat equation and the index theorem" Invent. Math. , 19 (1973) pp. 279--330.\\
\bibitem{Singer} M. Atiyah, I. Singer, "The index of elliptic operators IV" Ann. of Math. , 93 (1971) pp. 119--138. \\
\bibitem{Alvarez} L. Alvarez-Gaume, "Supersymmetry and the Atiyah Singer index theorem" Comm. Math. Phys. , 90 (1983) pp. 161--170.\\
\bibitem{Atiyah} Atiyah M. https://hitsmediaweb.h-its.org/Mediasite/Play/35600dda1dec419cb4e99f706197a3951d. \\ 
\bibitem{Eddington} Eddington A, Fundamental Theory, Cambridge.\\
\bibitem{Sanchez} F.M. Sanchez, V. Kotov, M. Grosmann, D. Weigel, R. Veysseyre, C. Bizouard, N. Flawisky, D. Gayral, L. Gueroult, Back to Cosmos.\\
\bibitem{Bastin} Bastin T. and Kilmister C.W., Combinatorial Physics (World Scientific, 1995).\\
\bibitem{Tanabashi} Tanabashi M. et al. (Particle Data Group), Phys. Rev. D98, 030001 (2018), and 2019 update.\\
\bibitem{Atiyah1} Atiyah M. Private Communication (december 2018).\\
\bibitem{Wyler} Wyler A., "L'espace symetrique du groupe des equations de Maxwell" C. R. Acad. Sc. Paris, t. 269, 743-745 (1969). Wyler A., C.R. Acad. Sci, Paris "Les groupes des potentiels de Coulomb et de Yukawa". C. R. Acad. Sc. Paris, t. 272, 186-188 (1971).\\
\bibitem{Conway} Conway, John Horton; Norton, Simon P. (1979). "Monstrous Moonshine". Bull. London Math. Soc. 11 (3): 308--339.\\
\bibitem{Borcherds} Borcherds, Richard (1992), "Monstrous Moonshine and Monstrous Lie Superalgebras", Invent. Math., 109: 405--444.\\
\bibitem{Sanchez1}  Sanchez F.M., Holic Principle, Entelechies, ANPA 16, Sept. 1995. Bowden K.G., 324--343.\\
\bibitem{Shannon} Shannon C.E. « A Mathematical Theory of Communication » Reprinted with corrections from The Bell System Technical Journal, Vol. 27, p. 379–423, 623–656, July, October, 1948.)\\
\bibitem{Stark} Stark H.M. A complete determination of the complex quadratic fields of class-number one, Michigan Math. J., vol. 14,‎ 1967, p. 1-27  \\
\bibitem{Lovelace} Lovelace C. (1971) Pomeron form factors and dual Regee cuts, Physics Letters B34 (6) 500-506.\\
\bibitem{Apostol} Apostol T. Modular functions and Dirichlet Series in Number Theory. Springler-Verlag. New-York (1990).\\
\bibitem{Green} Green, M. Schwarz J. (1984)  Anomaly cancellations in supersymmetric D = 10 gauge theory and superstring theory". Physics Letters B. 149: 117.\\
\bibitem{Shlay} Shray J. (1994) Octonions and Supersymmetry, PhD thesis.  http://ir.library.oregonstate.edu/xmlui/handle/1957/35649. \\
\bibitem{Koide} Koide Y., Fermion-Boson Two-Body Model of Quarks and Leptons and Cabibbo Mixing  Lett. Nuovo Cimento 34, 201 (1982). 
\bibitem{Hooft} Hooft 't Th Holographic Principle. ArXiv: hep-th/003004 (2000). \\
\bibitem{Bousso} Bousso R., The Holographic Principle, Review of Modern Physics, vol 74, p.834 (2002).\\
\bibitem{Friedman} Friedman W. et al, The Carnegie-Chicago Hubble Program. VIII. An Independent Determination of the Hubble Constant Based on the Tip of the Red Giant Branch, arxiv : 1907.05922.\\ 
\bibitem{Durham} Durham I.T. 2006, Sir Arthur Eddington and the Foundations of Modern Physics arXiv:quant-ph/0603146v1  p.111.
\bibitem{Sanchez2} Sanchez F.M., Kotov V. and Bizouard C., 'Towards a synthesis of two cosmologies: the steady- state flickering Universe'. Journal of Cosmology, vol 17. (2011).\\
\bibitem{Quinn} Quinn T, Speake C, Parks H, Davis R. 2014 The BIPM measurements of the Newtonian constant of gravitation, G. Phil.Trans. R. Soc. A372: 20140032. http://dx.doi.org/10.1098/rsta.2014.0032. \\
\bibitem{Sternheimer} Sternheimer J., Musique des particules elementaires, CRAS, 297, II, 829--834 (1983).\\
\bibitem{Weigel} Veysseyre R., Veysseyre H., and Weigel D. Counting, and Symbols of Cristallographic Point Symmetry Operations of Space En. AAECC 5, 53--70 (1994).\\
\bibitem{Carr} Carr B.J. and Rees M. J. , “The anthropic principle and the structure of the physical world”, Nature 278, 605-612 (1979).\\
\bibitem{Sanchez3} F.M. Sanchez. Coherent Cosmology Vixra.org,1601.0011. Springer International Publishing AG 2017. A. Tadjer et al. (eds.), Quantum Systems in Physics, Chemistry, and Biology, Progress in Theoretical Chemistry and Physics 30, pp. 375--407. \\ 

\end{thebibliography}
%
%%%%%%%%%%%%%%%%%%%%%%%%%%%%%%%%%%%%%%%%%%%%%%%%%%%%%%%%%%%%%%
Examples for figures using graphicx
A guide "Using Imported Graphics in LaTeX2e"  (Keith Reckdahl)
is available on a lot of LaTeX public servers or ctan mirrors.
The file is : epslatex.pdf 
%%%%%%%%%%%%%%%%%%%%%%%%%%%%%%%%%%%%%%%%%%%%%%%%%%%%%%%%%%%%%%

%-------------------------------------------------------------
%                 A figure as large as the width of the column
%-------------------------------------------------------------
   \begin{figure}
   \centering
   \includegraphics[width=\hsize]{empty.eps}
      \caption{Vibrational stability equation of state
               $S_{\mathrm{vib}}(\lg e, \lg \rho)$.
               $>0$ means vibrational stability.
              }
         \label{FigVibStab}
   \end{figure}
%
%-------------------------------------------------------------
%                                    One column rotated figure
%-------------------------------------------------------------
   \begin{figure}
   \centering
   \includegraphics[angle=-90,width=3cm]{empty.eps}
      \caption{Vibrational stability equation of state
               $S_{\mathrm{vib}}(\lg e, \lg \rho)$.
               $>0$ means vibrational stability.
              }
         \label{FigVibStab}
   \end{figure}
%
%-------------------------------------------------------------
%                        Figure with caption on the right side 
%-------------------------------------------------------------
   \begin{figure}
   \sidecaption
   \includegraphics[width=3cm]{empty.eps}
      \caption{Vibrational stability equation of state
               $S_{\mathrm{vib}}(\lg e, \lg \rho)$.
               $>0$ means vibrational stability.
              }
         \label{FigVibStab}
   \end{figure}
%
%-------------------------------------------------------------
%                                Figure with a new BoundingBox 
%-------------------------------------------------------------
   \begin{figure}
   \centering
   \includegraphics[bb=10 20 100 300,width=3cm,clip]{empty.eps}
      \caption{Vibrational stability equation of state
               $S_{\mathrm{vib}}(\lg e, \lg \rho)$.
               $>0$ means vibrational stability.
              }
         \label{FigVibStab}
   \end{figure}
%
%-------------------------------------------------------------
%                                      The "resizebox" command 
%-------------------------------------------------------------
   \begin{figure}
   \resizebox{\hsize}{!}
            {\includegraphics[bb=10 20 100 300,clip]{empty.eps}
      \caption{Vibrational stability equation of state
               $S_{\mathrm{vib}}(\lg e, \lg \rho)$.
               $>0$ means vibrational stability.
              }
         \label{FigVibStab}
   \end{figure}
%
%-------------------------------------------------------------
%                                             Two column Figure 
%-------------------------------------------------------------
   \begin{figure*}
   \resizebox{\hsize}{!}
            {\includegraphics[bb=10 20 100 300,clip]{empty.eps}
      \caption{Vibrational stability equation of state
               $S_{\mathrm{vib}}(\lg e, \lg \rho)$.
               $>0$ means vibrational stability.
              }
         \label{FigVibStab}
   \end{figure*}
%
%-------------------------------------------------------------
%                                             Simple A&A Table
%-------------------------------------------------------------
%
\begin{table}
\caption{Nonlinear Model Results}             % title of Table
\label{table:1}      % is used to refer this table in the text
\centering                          % used for centering table
\begin{tabular}{c c c c}        % centered columns (4 columns)
\hline\hline                 % inserts double horizontal lines
HJD & $E$ & Method\#2 & Method\#3 \\    % table heading 
\hline                        % inserts single horizontal line
   1 & 50 & $-837$ & 970 \\      % inserting body of the table
   2 & 47 & 877    & 230 \\
   3 & 31 & 25     & 415 \\
   4 & 35 & 144    & 2356 \\
   5 & 45 & 300    & 556 \\ 
\hline                                   %inserts single line
\end{tabular}
\end{table}
%
%-------------------------------------------------------------
%                                             Two column Table 
%-------------------------------------------------------------
%
\begin{table*}
\caption{Nonlinear Model Results}             
\label{table:1}      
\centering          
\begin{tabular}{c c c c l l l }     % 7 columns 
\hline\hline       
                      % To combine 4 columns into a single one 
HJD & $E$ & Method\#2 & \multicolumn{4}{c}{Method\#3}\\ 
\hline                    
   1 & 50 & $-837$ & 970 & 65 & 67 & 78\\  
   2 & 47 & 877    & 230 & 567& 55 & 78\\
   3 & 31 & 25     & 415 & 567& 55 & 78\\
   4 & 35 & 144    & 2356& 567& 55 & 78 \\
   5 & 45 & 300    & 556 & 567& 55 & 78\\
\hline                  
\end{tabular}
\end{table*}
%
%-------------------------------------------------------------
%                                          Table with notes 
%-------------------------------------------------------------
%
% A single note
\begin{table}
\caption{\label{t7}Spectral types and photometry for stars in the
  region.}
\centering
\begin{tabular}{lccc}
\hline\hline
Star&Spectral type&RA(J2000)&Dec(J2000)\\
\hline
69           &B1\,V     &09 15 54.046 & $-$50 00 26.67\\
49           &B0.7\,V   &*09 15 54.570& $-$50 00 03.90\\
LS~1267~(86) &O8\,V     &09 15 52.787&11.07\\
24.6         &7.58      &1.37 &0.20\\
\hline
LS~1262      &B0\,V     &09 15 05.17&11.17\\
MO 2-119     &B0.5\,V   &09 15 33.7 &11.74\\
LS~1269      &O8.5\,V   &09 15 56.60&10.85\\
\hline
\end{tabular}
\tablefoot{The top panel shows likely members of Pismis~11. The second
panel contains likely members of Alicante~5. The bottom panel
displays stars outside the clusters.}
\end{table}
%
% More notes
%
\begin{table}
\caption{\label{t7}Spectral types and photometry for stars in the
  region.}
\centering
\begin{tabular}{lccc}
\hline\hline
Star&Spectral type&RA(J2000)&Dec(J2000)\\
\hline
69           &B1\,V     &09 15 54.046 & $-$50 00 26.67\\
49           &B0.7\,V   &*09 15 54.570& $-$50 00 03.90\\
LS~1267~(86) &O8\,V     &09 15 52.787&11.07\tablefootmark{a}\\
24.6         &7.58\tablefootmark{1}&1.37\tablefootmark{a}   &0.20\tablefootmark{a}\\
\hline
LS~1262      &B0\,V     &09 15 05.17&11.17\tablefootmark{b}\\
MO 2-119     &B0.5\,V   &09 15 33.7 &11.74\tablefootmark{c}\\
LS~1269      &O8.5\,V   &09 15 56.60&10.85\tablefootmark{d}\\
\hline
\end{tabular}
\tablefoot{The top panel shows likely members of Pismis~11. The second
panel contains likely members of Alicante~5. The bottom panel
displays stars outside the clusters.\\
\tablefoottext{a}{Photometry for MF13, LS~1267 and HD~80077 from
Dupont et al.}
\tablefoottext{b}{Photometry for LS~1262, LS~1269 from
Durand et al.}
\tablefoottext{c}{Photometry for MO2-119 from
Mathieu et al.}
}
\end{table}
%
%-------------------------------------------------------------
%                                       Table with references 
%-------------------------------------------------------------
%
\begin{table*}[h]
 \caption[]{\label{nearbylistaa2}List of nearby SNe used in this work.}
\begin{tabular}{lccc}
 \hline \hline
  SN name &
  Epoch &
 Bands &
  References \\
 &
  (with respect to $B$ maximum) &
 &
 \\ \hline
1981B   & 0 & {\it UBV} & 1\\
1986G   &  $-$3, $-$1, 0, 1, 2 & {\it BV}  & 2\\
1989B   & $-$5, $-$1, 0, 3, 5 & {\it UBVRI}  & 3, 4\\
1990N   & 2, 7 & {\it UBVRI}  & 5\\
1991M   & 3 & {\it VRI}  & 6\\
\hline
\noalign{\smallskip}
\multicolumn{4}{c}{ SNe 91bg-like} \\
\noalign{\smallskip}
\hline
1991bg   & 1, 2 & {\it BVRI}  & 7\\
1999by   & $-$5, $-$4, $-$3, 3, 4, 5 & {\it UBVRI}  & 8\\
\hline
\noalign{\smallskip}
\multicolumn{4}{c}{ SNe 91T-like} \\
\noalign{\smallskip}
\hline
1991T   & $-$3, 0 & {\it UBVRI}  &  9, 10\\
2000cx  & $-$3, $-$2, 0, 1, 5 & {\it UBVRI}  & 11\\ %
\hline
\end{tabular}
\tablebib{(1)~\citet{branch83};
(2) \citet{phillips87}; (3) \citet{barbon90}; (4) \citet{wells94};
(5) \citet{mazzali93}; (6) \citet{gomez98}; (7) \citet{kirshner93};
(8) \citet{patat96}; (9) \citet{salvo01}; (10) \citet{branch03};
(11) \citet{jha99}.
}
\end{table}
%-------------------------------------------------------------
%                      A rotated Two column Table in landscape  
%-------------------------------------------------------------
\begin{sidewaystable*}
\caption{Summary for ISOCAM sources with mid-IR excess 
(YSO candidates).}\label{YSOtable}
\centering
\begin{tabular}{crrlcl} 
\hline\hline             
ISO-L1551 & $F_{6.7}$~[mJy] & $\alpha_{6.7-14.3}$ 
& YSO type$^{d}$ & Status & Comments\\
\hline
  \multicolumn{6}{c}{\it New YSO candidates}\\ % To combine 6 columns into a single one
\hline
  1 & 1.56 $\pm$ 0.47 & --    & Class II$^{c}$ & New & Mid\\
  2 & 0.79:           & 0.97: & Class II ?     & New & \\
  3 & 4.95 $\pm$ 0.68 & 3.18  & Class II / III & New & \\
  5 & 1.44 $\pm$ 0.33 & 1.88  & Class II       & New & \\
\hline
  \multicolumn{6}{c}{\it Previously known YSOs} \\
\hline
  61 & 0.89 $\pm$ 0.58 & 1.77 & Class I & \object{HH 30} & Circumstellar disk\\
  96 & 38.34 $\pm$ 0.71 & 37.5& Class II& MHO 5          & Spectral type\\
\hline
\end{tabular}
\end{sidewaystable*}
%-------------------------------------------------------------
%                      A rotated One column Table in landscape  
%-------------------------------------------------------------
\begin{sidewaystable}
\caption{Summary for ISOCAM sources with mid-IR excess 
(YSO candidates).}\label{YSOtable}
\centering
\begin{tabular}{crrlcl} 
\hline\hline             
ISO-L1551 & $F_{6.7}$~[mJy] & $\alpha_{6.7-14.3}$ 
& YSO type$^{d}$ & Status & Comments\\
\hline
  \multicolumn{6}{c}{\it New YSO candidates}\\ % To combine 6 columns into a single one
\hline
  1 & 1.56 $\pm$ 0.47 & --    & Class II$^{c}$ & New & Mid\\
  2 & 0.79:           & 0.97: & Class II ?     & New & \\
  3 & 4.95 $\pm$ 0.68 & 3.18  & Class II / III & New & \\
  5 & 1.44 $\pm$ 0.33 & 1.88  & Class II       & New & \\
\hline
  \multicolumn{6}{c}{\it Previously known YSOs} \\
\hline
  61 & 0.89 $\pm$ 0.58 & 1.77 & Class I & \object{HH 30} & Circumstellar disk\\
  96 & 38.34 $\pm$ 0.71 & 37.5& Class II& MHO 5          & Spectral type\\
\hline
\end{tabular}
\end{sidewaystable}
%
%-------------------------------------------------------------
%                              Table longer than a single page  
%-------------------------------------------------------------
% All long tables will be placed automatically at the end of the document
%
\longtab{
\begin{longtable}{lllrrr}
\caption{\label{kstars} Sample stars with absolute magnitude}\\
\hline\hline
Catalogue& $M_{V}$ & Spectral & Distance & Mode & Count Rate \\
\hline
\endfirsthead
\caption{continued.}\\
\hline\hline
Catalogue& $M_{V}$ & Spectral & Distance & Mode & Count Rate \\
\hline
\endhead
\hline
\endfoot
%%
Gl 33    & 6.37 & K2 V & 7.46 & S & 0.043170\\
Gl 66AB  & 6.26 & K2 V & 8.15 & S & 0.260478\\
Gl 68    & 5.87 & K1 V & 7.47 & P & 0.026610\\
         &      &      &      & H & 0.008686\\
Gl 86 
\footnote{Source not included in the HRI catalog. See Sect.~5.4.2 for details.}
         & 5.92 & K0 V & 10.91& S & 0.058230\\
\end{longtable}
}
%
%-------------------------------------------------------------
%                              Table longer than a single page
%                                            and in landscape, 
%                    in the preamble, use: \usepackage{lscape}
%-------------------------------------------------------------

% All long tables will be placed automatically at the end of the document
%
\longtab{
\begin{landscape}
\begin{longtable}{lllrrr}
\caption{\label{kstars} Sample stars with absolute magnitude}\\
\hline\hline
Catalogue& $M_{V}$ & Spectral & Distance & Mode & Count Rate \\
\hline
\endfirsthead
\caption{continued.}\\
\hline\hline
Catalogue& $M_{V}$ & Spectral & Distance & Mode & Count Rate \\
\hline
\endhead
\hline
\endfoot
%%
Gl 33    & 6.37 & K2 V & 7.46 & S & 0.043170\\
Gl 66AB  & 6.26 & K2 V & 8.15 & S & 0.260478\\
Gl 68    & 5.87 & K1 V & 7.47 & P & 0.026610\\
         &      &      &      & H & 0.008686\\
Gl 86
\footnote{Source not included in the HRI catalog. See Sect.~5.4.2 for details.}
         & 5.92 & K0 V & 10.91& S & 0.058230\\
\end{longtable}
\end{landscape}
}
%
%-------------------------------------------------------------
%               Appendices have to be placed at the end, after
%                                        \end{thebibliography}
%-------------------------------------------------------------
\end{thebibliography}

\begin{appendix} %First appendix
\section{Background galaxy number counts and shear noise-levels}
Because the optical images used in this analysis...
\begin{figure*}%f1
\includegraphics[width=10.9cm]{1787f23.eps}
\caption{Shown in greyscale is a...}
\label{cl12301}
\end{figure*}

In this case....
\begin{figure*}
\centering
\includegraphics[width=16.4cm,clip]{1787f24.ps}
\caption{Plotted above...}
\label{appfig}
\end{figure*}

Because the optical images...

\section{Title of Second appendix.....} %Second appendix
These studies, however, have faced...
\begin{table}
\caption{Complexes characterisation.}\label{starbursts}
\centering
\begin{tabular}{lccc}
\hline \hline
Complex & $F_{60}$ & 8.6 &  No. of  \\
...
\hline
\end{tabular}
\end{table}

The second method produces...
\end{appendix}
%
%
\end{document}

%
%-------------------------------------------------------------
%          For the appendices, table longer than a single page
%-------------------------------------------------------------

% Table will be print automatically at the end of the document, 
% after the whole appendices

\begin{appendix} %First appendix
\section{Background galaxy number counts and shear noise-levels}

% In the appendices do not forget to put the counter of the table 
% as an option

\longtab[1]{
\begin{longtable}{lrcrrrrrrrrl}
\caption{Line data and abundances ...}\\
\hline
\hline
Def & mol & Ion & $\lambda$ & $\chi$ & $\log gf$ & N & e &  rad & $\delta$ & $\delta$ 
red & References \\
\hline
\endfirsthead
\caption{Continued.} \\
\hline
Def & mol & Ion & $\lambda$ & $\chi$ & $\log gf$ & B & C &  rad & $\delta$ & $\delta$ 
red & References \\
\hline
\endhead
\hline
\endfoot
\hline
\endlastfoot
A & CH & 1 &3638 & 0.002 & $-$2.551 &  &  &  & $-$150 & 150 &  Jorgensen et al. (1996) \\                    
\end{longtable}
}% End longtab
\end{appendix}

%-------------------------------------------------------------
%                   For appendices and landscape, large table:
%                    in the preamble, use: \usepackage{lscape}
%-------------------------------------------------------------

\begin{appendix} %First appendix
%
\longtab[1]{
\begin{landscape}
\begin{longtable}{lrcrrrrrrrrl}
...
\end{longtable}
\end{landscape}
}% End longtab
\end{appendix}

%%%% End of aa.dem
